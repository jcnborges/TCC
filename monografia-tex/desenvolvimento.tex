%---------- Segundo Capitulo ----------
\chapter{Desenvolvimento}
\label{chap:desenv}

\section{Placa de Roteamento}
\label{sec:desroteamento}

A placa de roteamento � um componente fundamental do projeto, respons�vel por realizar a interliga��o entre a placa C8051F340 e os componentes de hardware do rob�. A equipe constatou a necessidade deste componente devido aos seguintes fatores:

\begin{itemize}
    \item[-] Necessidade de alimenta��o dedicada para alguns componentes
    \item[-] Roteamento das leituras de cada sensor para o pino I/O correto da C8051F340
    \item[-] Necessidade de um buffer para o PWM
    \item[-] Melhor organiza��o do rob�
\end{itemize}

Como descrito nas especifica��es do rob� (\ref{chap:esprob}, o rob� Bellator possui, dentre outros componentes, cinco sensores, dois encoders e duas pontes H. Os cinco sensores e dois encoders necessitam de alimenta��o de aproximadamente 5 Volts para opera��o, e as duas pontes H necessitam de um sinal de PWM de baixa imped�ncia. #TODO - Explicar consumo de corrente e alimenta��o.

Al�m disso, a dificuldade de conectar, de forma pr�tica, todos os componentes do rob� com a C8051F340 bem como a quantidade excessiva de fios resultante fez com que a equipe conclu�sse que a placa de roteamento � indispens�vel.

De acordo com os fatores descritos, a equipe construiu uma lista de requisitos para a placa de roteamento:

\begin{itemize}
    \item[-] Fornecer alimenta��o de aproximadamente 5 Volts
    \item[-] Capacidade de corrente suficiente %Atualizar com valores!
    \item[-] Conectores pr�ticos para interface com a C8051F340 e o resto do rob�
    \item[-] Fornecer um buffer para o sinal de PWM
\end{itemize}


%Descrever os fatores. Escrever requisitos. Descrever placa. Explicar a produ��o da placa. 