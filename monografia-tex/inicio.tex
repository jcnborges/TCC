% ---------- Preambulo ----------
\instituicao{Universidade Tecnol\'ogica Federal do Paran\'a} % nome da instituicao
\departamento{Departamento Acad\^emico de Eletr\^onica} % nome do departamento
\programa{Curso Superior de Tecnologia em Mecatr\^onica Industrial} % nome do curso

\documento{Trabalho de Conclus\~ao de Curso} % [Trabalho de Conclus\~ao de Curso] ou [Relat\'orio de Est\'agio]
\titulacao{Tecn\'ologo} % [T\'ecnico], [Tecn\'ologo] ou [Engenheiro]

\titulo{T\'itulo em Portugu\^es} % titulo do trabalho em portugues
\title{Title in English} % titulo do trabalho em ingles

\autor{Nome do Primeiro Autor J�lio} % primeiro autor do trabalho
\autordois{Nome do Segundo Autor} % segundo autor do trabalho, caso exista
%\autortres{Nome do Terceiro Autor} % terceiro autor do trabalho, caso exista
%\autorquatro{Nome do Quarto Autor} % quarto autor do trabalho, caso exista
\cita{SOBRENOME 1, Nome 1; SOBRENOME 2, Nome 2} % sobrenome (maiusculas) e nome do(s) autor(es) do trabalho, separados por ponto-e-virgula (ate quatro autores para TCC)

\palavraschave{Palavra-chave 1, Palavra-chave 2, ...} % palavras-chave do trabalho
\keywords{Keyword 1, Keyword 2, ...} % palavras-chave do trabalho em ingles

\comentario{\UTFPRdocumentodata\ apresentado ao \UTFPRdepartamentodata\ como requisito parcial para obten\c{c}\~ao do grau de \UTFPRtitulacaodata\ no \UTFPRprogramadata\ da \ABNTinstituicaodata.}

\orientador{Nome do Orientador} % nome do orientador do trabalho
%\orientador[Orientadora:]{Nome da Orientadora} % <- no caso de orientadora, usar esta sintaxe
\coorientador{Nome do Co-orientador} % nome do co-orientador do trabalho, caso exista
%\coorientador[Co-orientadora:]{Nome da Co-orientadora} % <- no caso de co-orientadora, usar esta sintaxe

\local{Curitiba} % cidade
\data{2009} % ano


%---------- Inicio do Documento ----------
\begin{document}

\capa % geracao automatica da capa
\folhaderosto % geracao automatica da folha de rosto
%\termodeaprovacao % <- ainda a ser implementado corretamente

% dedicat�ria (opcional)
\begin{dedicatoria}
Texto da dedicat\'oria.
\end{dedicatoria}

% agradecimentos (opcional)
\begin{agradecimentos}
Texto dos agradecimentos.
\end{agradecimentos}

% epigrafe (opcional)
\begin{epigrafe}
Texto da ep\'igrafe.
\end{epigrafe}

%resumo
\begin{resumo}
Texto do resumo (m\'aximo de 500 palavras).
\end{resumo}

%abstract
\begin{abstract}
Abstract text (maximum of 500 words).
\end{abstract}

% listas (opcionais, mas recomenda-se a partir de 5 elementos)
\listadefiguras % geracao automatica da lista de figuras
\listadetabelas % geracao automatica da lista de tabelas
\listadesiglas % geracao automatica da lista de siglas
\listadesimbolos % geracao automatica da lista de simbolos

% sumario
\sumario % geracao automatica do sumario


%---------- Inicio do Texto ----------
% recomenda-se a escrita de cada capitulo em um arquivo texto separado (exemplo: intro.tex, fund.tex, exper.tex, concl.tex, etc.) e a posterior inclusao dos mesmos no mestre do documento utilizando o comando \input{}, da seguinte forma:
%\input{intro.tex}
%\input{fund.tex}
%\input{exper.tex}
%\chapter{Conclus�o}

O presente trabalho de conclus�o de curso apresentou objetivos abrangentes, envolvendo desenvolvimento de hardware e software. No contexto da navega��o rob�tica, surgiu a necessidade de se utilizar um rob� real com a finalidade de se obter resultados mais significativos. Desse modo, foi elaborado um projeto cujo escopo tamb�m foi reconstruir e adequar uma plataforma rob�tica previamente dispon�vel, que � descrita na sec��o \ref{sec:estpro}, por�m que n�o estava em condi��es de uso imediato. A equipe procedeu com testes em laborat�rio de eletr�nica afim de avaliar as condi��es iniciais do rob�, conforme � descrito na sec��o \ref{sec:testecomp}. Uma an�lise de software foi efetuada e o c�digo original do microcontrolador C8051F340DK, disponibilizado como parte integrante do rob� Bellator \cite{BELLATOR}, foi avaliado e reconfigurado de acordo com as necessidades do projeto, o que � descrito em detalhes na se��o \ref{sec:codmicro}. Havendo necessidade de implementa��o de hardware, a equipe projetou e construiu uma placa de roteamento para alimentar os sensores e encoders, assim como tratar os sinais destes e os de PWM, como � descrito na se��o \ref{sec:desroteamento}. Finalmente, o hardware acoplado foi configurado e o software que executou os algortimos de navega��o foi desenvolvido, conforme a se��o \ref{sec:ts}. Com isso, a equipe concluiu a primeira parte do projeto, a qual consistiu na reconstru��o e adequa��o do rob� Bellator.

Estando a plataforma rob�tica funcional, seguiram-se o projeto e implementa��o dos algoritmos de navega��o propostos. A L�gica Fuzzy foi estudada e apresentada na se��o \ref{sec:logfuzzy} e a metodologia de Mapas Cognitivos Fuzzy (FCM) foi estudada e apresentada na se��o \ref{sec:fcm}. Com a teoria fundamentada, a equipe projetou os algoritmos e implementou-os na linguagem C++, compilados para executar no hardware acoplado, placa TS-7260. O projeto e a implementa��o dos algoritmos de navega��o Fuzzy e FCM s�o descritos em detalhes nas se��es \ref{sec:algfuzzy} e \ref{sec:algedfcm}, respectivamente. Ap�s a implementa��o, a equipe submeteu os algoritmos a uma s�rie de testes b�sicos, denominada Testes Iniciais, que serviram para fornecer a primeira realimenta��o do projeto dos algoritmos e pode ser lido na se��o \ref{sec:testesini}. Finalizados esses testes, foram elaborados testes complexos, denominados Testes Avan�ados, para estressar os sistemas de navega��o propostos e fornecer a segunda realimenta��o do projeto dos algoritmos, conforme � descri��o na se��o \ref{sec:testesavan}. Finalmente, ap�s os testes avan�ados, os algoritmos resolviam problemas complexos de navega��o, como o corredor sem sa�da e o problema de decis�o quando dois obst�culos laterais e um frontal era colocado diante do rob�, e poderiam ser usados nos testes finais, que forneceram os dados para a an�lise de resultados e foram denominados Testes Comparativos, conforme � descrito na se��o \ref{sec:testescomp}. Com isso foi conclu�do outros dois objetivos do projeto, que foram o projeto e implementa��o dos algoritmos de navega��o e a elabora��o e execu��o de uma metodologia de testes comparativos entre os algoritmos.

Para trabalhos futuros utilizando a plataforma Bellator reconstru�da, a equipe recomenda combinar sensores de ultra-som ao sensores infra vermelho, justificando isso porque os sensores de ultra-som apresentam uma faixa de opera��o cuja dist�ncia m�nima � menor que a do infra-vermelho, podendo capturar dist�ncias de 2 cm. Atualmente a dist�ncia m�nima suportada pelo sistema de navega��o � de 15 cm, com os sensores de ultra-som, os algoritmos poderiam operar em uma faixa mais abrangente. Outra sugest�o � introduzir ao sistema uma realimenta��o por b�ssola pois nesse projeto a realimenta��o odom�trica fornecida pelos encoders � utilizada para ajustar a velocidade das rodas e n�o faz uma interpreta��o da dire��o do rob�. Para tornar o rob� seguro para o manuseio, sugere-se a fixa��o dos sensores parafusando-os no chassi do Bellator e acoplando uma casca que proteja os circuitos microcontrolados. A equipe tamb�m recomenda a reconstru��o da placa de roteamento utilizando um m�todo industrial para confecc��o de placas de circuito impresso. Para os sistemas de navega��o, um trabalho futuro de grande riqueza seria introduzir ao sistema a capacidade de interpretar a posi��o do rob� em rela��o a um referencial. Com isso, o rob� seria capaz de resolver problemas nos quais este deve partir de um ponto inicial no espa�o a um ponto final, guiando-se pelos sensores para evitar colis�es e realimentar-se por um sistema de posicionamento para corrigir a traget�ria. Outro trabalho, produto deste, seria introduzir ao sistema uma mem�ria a qual pudesse mapear os obst�culos capturados pelos sensores do rob�, assim sendo, produzir-se-ia um artefato aut�nomo capaz de mapear terrenos. Outro aprimoramento da plataforma seria implementar um sistema de controle remoto por joystick, na qual uma base remota podesse pilotar o rob� via joystick. Finalmente, a equipe sugere um projeto futuro no qual seja implementado um sistema de vis�o computacional por c�mera de v�deo.

A equipe concluiu esta monografia justificando que os objetivos descritos na introdu��o, se��o \ref{rec:obj}, foram alcan�ados e est�o de acordo com os requisitos m�nimos de um curso de Engenharia de Computa��o. Os problemas encontrados na execu��o do projeto est�o associados ao escopo abrangente do mesmo, o qual envolveu o desenvolvimento de hardware e software em um projeto integrador. A subdivis�o do projeto em diversos objetivos, sendo um pr�-requisito para o outro, foi inevit�vel para alcan�ar os resultados finais. A equipe encontrou dificuldades durante a reconstru��o do rob�, configura��o da placa TS-7260, testes integrados de funcionamento do rob�, implementa��o dos algoritmos, execu��o dos testes dos algoritmos e an�lise de resultados. Durante a recontru��o, os componentes eletr�nicos foram testados isoladamente e houve riscos de haver danos, o que representaria atrasos no projeto. A placa TS-7260 apresentou complexidade para ser configurada pois n�o houve um t�cnico dispon�vel para auxiliar a equipe, a qual teve que aprender a trabalhar com esse hardware. Nos testes de integra��o da C8051F340DK e da TS-7260, que determinaram o funcionamento da plataforma rob�tica, foi exigido processos de depura��o integrados, nos quais os problemas foram isolados e corrigidos repetidas vezes. A implementa��o dos algoritmos at� a vers�o final, que foi utilizada nos testes comparativos, foi realizada paralelamente aos testes b�sicos e avan�ados, nas quais os problemas de navega��o foram detectados, isolados e corrigidos repetidas vezes. A metodologia de testes escolhida foi elaborada pela equipe e foram efetuados v�rios experimentos com registro em v�deo at� que se atingessem os resultados finais. Tendo com base os v�deos gravados e a experi�ncia em campo observada, a equipe precisou analisar os resultados, discut�-los e extrair conclus�es para finalizar o projeto, sendo que essa tarefa representou um trabalho cient�fico. 

