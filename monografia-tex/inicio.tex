% ---------- Preambulo ----------
\instituicao{Universidade Tecnol�gica Federal do Paran�} % nome da instituicao
\departamento{Departamento Acad�mico de Inform�tica} % nome do departamento
\programa{Bacharelado em Engenharia de Computa��o} % nome do curso

\documento{Trabalho de Conclus�o de Curso} % [Trabalho de Conclus\~ao de Curso] ou [Relat\'orio de Est\'agio]
\titulacao{Engenheiro} % [T\'ecnico], [Tecn\'ologo] ou [Engenheiro]

\titulo{An�lise Qualitativa de Algoritmos de Navega��o Fuzzy} % titulo do trabalho em portugues
\title{Qualitative Analysis of Fuzzy Algorithms} % titulo do trabalho em ingles

\autor{Alexandre Jacques Marin} % primeiro autor do trabalho
\autordois{J�lio Cesar Nardelli Borges} % segundo autor do trabalho, caso exista
\autortres{Yuri Antin Wergrzn} % terceiro autor do trabalho, caso exista
%\autorquatro{Nome do Quarto Autor} % quarto autor do trabalho, caso exista
\cita{MARIN, Alexandre; BORGES, J�lio; WERGRZN, Yuri } % sobrenome (maiusculas) e nome do(s) autor(es) do trabalho, separados por ponto-e-virgula (ate quatro autores para TCC)

\palavraschave{Navega��o, Fuzzy, Rob�s, Aut�noma, ...} % palavras-chave do trabalho
\keywords{Navigation, Fuzzy, Robots, Autonomous, ...} % palavras-chave do trabalho em ingles

\comentario{\UTFPRdocumentodata\ apresentado ao \UTFPRdepartamentodata\ como requisito parcial para obten\c{c}\~ao do grau de \UTFPRtitulacaodata\ no \UTFPRprogramadata\ da \ABNTinstituicaodata.}

\orientador{Jo�o Alberto Fabro} % nome do orientador do trabalho
%\orientador[Orientadora:]{Nome da Orientadora} % <- no caso de orientadora, usar esta sintaxe
\coorientador{Heitor Silv�rio Lopes} % nome do co-orientador do trabalho, caso exista
%\coorientador[Co-orientadora:]{Nome da Co-orientadora} % <- no caso de co-orientadora, usar esta sintaxe

\local{Curitiba} % cidade
\data{2012} % ano


%---------- Inicio do Documento ----------
\begin{document}

\capa % geracao automatica da capa
\folhaderosto % geracao automatica da folha de rosto
%\termodeaprovacao % <- ainda a ser implementado corretamente

% dedicat�ria (opcional)
\begin{dedicatoria}
Texto da dedicat\'oria.
\end{dedicatoria}

% agradecimentos (opcional)
\begin{agradecimentos}
Texto dos agradecimentos.
\end{agradecimentos}

% epigrafe (opcional)
\begin{epigrafe}
Texto da ep\'igrafe.
\end{epigrafe}

%resumo
\begin{resumo}
Este documento descreve detalhadamente a execu��o do projeto \"An�lise Qualitativa de Algoritmos de Navega��o Fuzzy\", feito como trabalho de conclus�o de curso de Engenharia de Computa��o na Universidade Tecnol�gica Federal do Paran�.
\end{resumo}

%abstract
\begin{abstract}
Abstract text (maximum of 500 words).
\end{abstract}

% listas (opcionais, mas recomenda-se a partir de 5 elementos)
\listadefiguras % geracao automatica da lista de figuras
\listadetabelas % geracao automatica da lista de tabelas
\listadesiglas % geracao automatica da lista de siglas
\listadesimbolos % geracao automatica da lista de simbolos

% sumario
\sumario % geracao automatica do sumario
