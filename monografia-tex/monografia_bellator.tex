% estilo de documento abnTeX
\documentclass{abnt-UTFPR}

% estilo de bibliografia abnTeX
\usepackage[alf,abnt-emphasize=bf,bibjustif,recuo=0cm]{abntcite}	

\usepackage[brazil]{babel}							% pacote portugues brasileiro
\usepackage[latin1]{inputenc}						% pacote para acentuacao direta
\usepackage{amsmath,amsfonts,amssymb}				% pacote matematico
\usepackage{float}
\usepackage{graphicx}								% pacote grafico
\usepackage{pgf}									% Pacote grafico
\usepackage{tikz}									% Pacote para desenhar figuras
\usepackage{times}									% fonte times
\usepackage{verbatim}								% multi-line comment
\usetikzlibrary{arrows, backgrounds, fit, petri}	% Modulos tikz
\usetikzlibrary{positioning, shapes}				% Modulos tikz

% ---------- Preambulo ----------
\instituicao{Universidade Tecnol\'ogica Federal do Paran\'a} % nome da instituicao
\departamento{Departamento Acad\^emico de Eletr\^onica} % nome do departamento
\programa{Curso Superior de Tecnologia em Mecatr\^onica Industrial} % nome do curso

\documento{Trabalho de Conclus\~ao de Curso} % [Trabalho de Conclus\~ao de Curso] ou [Relat\'orio de Est\'agio]
\titulacao{Tecn\'ologo} % [T\'ecnico], [Tecn\'ologo] ou [Engenheiro]

\titulo{T\'itulo em Portugu\^es} % titulo do trabalho em portugues
\title{Title in English} % titulo do trabalho em ingles

\autor{Nome do Primeiro Autor J�lio} % primeiro autor do trabalho
\autordois{Nome do Segundo Autor} % segundo autor do trabalho, caso exista
%\autortres{Nome do Terceiro Autor} % terceiro autor do trabalho, caso exista
%\autorquatro{Nome do Quarto Autor} % quarto autor do trabalho, caso exista
\cita{SOBRENOME 1, Nome 1; SOBRENOME 2, Nome 2} % sobrenome (maiusculas) e nome do(s) autor(es) do trabalho, separados por ponto-e-virgula (ate quatro autores para TCC)

\palavraschave{Palavra-chave 1, Palavra-chave 2, ...} % palavras-chave do trabalho
\keywords{Keyword 1, Keyword 2, ...} % palavras-chave do trabalho em ingles

\comentario{\UTFPRdocumentodata\ apresentado ao \UTFPRdepartamentodata\ como requisito parcial para obten\c{c}\~ao do grau de \UTFPRtitulacaodata\ no \UTFPRprogramadata\ da \ABNTinstituicaodata.}

\orientador{Nome do Orientador} % nome do orientador do trabalho
%\orientador[Orientadora:]{Nome da Orientadora} % <- no caso de orientadora, usar esta sintaxe
\coorientador{Nome do Co-orientador} % nome do co-orientador do trabalho, caso exista
%\coorientador[Co-orientadora:]{Nome da Co-orientadora} % <- no caso de co-orientadora, usar esta sintaxe

\local{Curitiba} % cidade
\data{2009} % ano


%---------- Inicio do Documento ----------
\begin{document}

\capa % geracao automatica da capa
\folhaderosto % geracao automatica da folha de rosto
%\termodeaprovacao % <- ainda a ser implementado corretamente

% dedicat�ria (opcional)
\begin{dedicatoria}
Texto da dedicat\'oria.
\end{dedicatoria}

% agradecimentos (opcional)
\begin{agradecimentos}
Texto dos agradecimentos.
\end{agradecimentos}

% epigrafe (opcional)
\begin{epigrafe}
Texto da ep\'igrafe.
\end{epigrafe}

%resumo
\begin{resumo}
Texto do resumo (m\'aximo de 500 palavras).
\end{resumo}

%abstract
\begin{abstract}
Abstract text (maximum of 500 words).
\end{abstract}

% listas (opcionais, mas recomenda-se a partir de 5 elementos)
\listadefiguras % geracao automatica da lista de figuras
\listadetabelas % geracao automatica da lista de tabelas
\listadesiglas % geracao automatica da lista de siglas
\listadesimbolos % geracao automatica da lista de simbolos

% sumario
\sumario % geracao automatica do sumario


%---------- Inicio do Texto ----------
% recomenda-se a escrita de cada capitulo em um arquivo texto separado (exemplo: intro.tex, fund.tex, exper.tex, concl.tex, etc.) e a posterior inclusao dos mesmos no mestre do documento utilizando o comando \input{}, da seguinte forma:
%\input{intro.tex}
%\input{fund.tex}
%\input{exper.tex}
%\chapter{Conclus�o}

O presente trabalho de conclus�o de curso apresentou objetivos abrangentes envolvendo desenvolvimento de hardware e software. No contexto da navega��o rob�tica, surgiu a necessidade de se utilizar um rob� real com a finalidade de se obterem resultados mais significativos. Desse modo, foi elaborado um projeto cujo escopo foi reconstruir e adequar uma plataforma previamente dispon�vel, que � descrita na sec��o \ref{sec:estpro}, por�m sem condi��es de uso imediato. A equipe procedeu com testes em laborat�rio de eletr�nica afim de avaliar as condi��es iniciais do rob�, conforme � descrito na sec��o \ref{sec:testecomp}. Uma an�lise de software foi efetuada e o c�digo original do microcontrolador C8051F340DK, disponibilizado como parte integrante do rob�, foi avaliado e reconfigurado de acordo com as necessidades do projeto, o que � descrito em detalhes na se��o \ref{sec:codmicro}. Havendo necessidade de implementa��o de hardware, a equipe projetou e construiu uma placa de roteamento para alimentar os sensores e encoders, assim como tratar os sinais destes e os de PWM, como � descrito na sec��o \ref{sec:desroteamento}. Finalmente, o hardware acoplado foi configurado e o software que executou os algortimos de navega��o foi desenvolvido, conforme a sec��o {sec:ts}. Com isso, a equipe concluiu a primeira parte do projeto, a qual consistiu na reconstru��o e adequa��o do rob� Bellator.

Estando a plataforma rob�tica funcional, procedeu-se ao projeto e implementa��o dos algoritmos de navega��o propostos. A L�gica Fuzzy foi estudada e apresentada na sec��o \ref{sec:logfuzzy} e a metodologia de Mapas Cognitivos Fuzzy (FCM) foi estudada e apresentada na sec��o \ref{sec:fcm}. Com a teoria fundamentada, a equipe projetou os algoritmos e implementou estes para executar na plataforma do projeto. Os c�digos foram escritos na linguagem C++ e compilados para executar no hardware da TS-7260. O projeto e a implementa��o dos algoritmos de navega��o fuzzy e FCM s�o descritos em detalhes nas sec��es \ref{sec:sec:algfuzzy} e \ref{sec:algedfcm}, respectivamente. Ap�s a implementa��o, a equipe submeteu os algoritmos a uma s�rie de testes b�sicos, denominada Testes Iniciais, que serviram como uma realimenta��o inicial do projeto dos algoritmos e pode ser lido na sec��o \ref{sec:testesini}. Finalizados esses testes, foram elaborados testes complexos, denominados Testes Avan�ados, para estressar os sistemas de navega��o propostos e fornecer outra realimenta��o do projeto dos algoritmos, conforme � descri��o na sec��o \ref{sec:testesavan}. Finalmente, ap�s os testes avan�ados, os algoritmos resolviam problemas complexos de navega��o, como o circuito em ``U" e o problema de decis�o quando dois obst�culos laterais e um frontal era colocado diante do rob�, e poderiam ser usados nos testes finais, que forneceram os dados para a an�lise de resultados e foram denominados Testes Comparativos, conforme � descrito na sec��o \ref{sec:testescomp}. Com isso foi conclu�do outros dois objetivos do projeto, que foram o projeto e implementa��o dos algoritmos de navega��o e a elabora��o e execu��o de uma metodologia de testes comparativos entre os algoritmos.

Para trabalhos futuros utilizando a plataforma Bellator reconstru�da, a equipe recomenda combinar sensores de ultra-som ao sensores infra vermelho, justificando isso porque os sensores de ultra-som apresentam uma faixa de opera��o cuja dist�ncia m�nima � menor que a do infra-vermelho, podendo capturar dist�ncia 2 cm. Desse modo, os algoritmos poderiam operar em uma faixa mais abrangente. Atualmente a dist�ncia m�nima suportada pelo sistema de navega��o � de 15 cm. Outra sugest�o � introduzir ao sistema uma realimenta��o por b�ssola pois nesse projeto a realimenta��o odom�trica fornecida pelos encoders � utilizada para ajustar a velocidade das rodas e n�o faz uma interpreta��o da dire��o do rob�. Para tornar o rob� seguro para o manuseio, sugere-se a fixa��o dos sensores parafusando-os no chassi do Bellator e acoplar uma casca que proteja os circuitos microcontrolados. A equipe tamb�m remenda a reconstru��o da placa de roteamento utilizando um m�todo industrial para confecc��o de placas de circuito impresso. Para os sistemas de navega��o, um trabalho futuro de grande riqueza seria introduzir ao sistema a capacidade de interpretar a posi��o do rob� em rela��o a um referencial. Com isso, o rob� seria capaz de resolver problemas nos quais este deve partir de um ponto inicial no espa�o a um ponto final, guiando-se pelos sensores para evitar colis�es e realimentar-se por um sistema de posicionamento para corrigir a traget�ria. Outro trabalho, produto deste, seria introduzir ao sistema uma mem�ria a qual pudesse mapear os obst�culos capturados pelos sensores do rob�, assim sendo, produzir-se-ia um artefato aut�nomo capaz de mapear terrenos.

Por fim, a equipe concluiu esta monografia justificando que os objetivos descritos na introdu��o, sec��o \ref{rec:obj}, foram alcan�ados e est�o de acordo com os requisitos m�nimos de um curso de Engenharia de Computa��o. Os problemas encontrados na execu��o do projeto est�o associados ao escopo abrangente do mesmo, o qual envolveu o desenvolvimento de hardware e software em um projeto integrado. A subdivis�o do projeto em diversos objetivos, sendo que um foi pr�-requisito para a execu��o do outro em uma execu��o encadeada, foi inevit�vel para alcan�ar os resultados finais. A equipe encontrou dificuldades em todas as etapas do projeto, desde a reconstru��o do rob� at� os testes avan�ados e an�lise de resultados. Os componentes eletr�nicos foram testados isoladamente e houve o riscos de haver danos, o que representaria atrasos no projeto. A placa TS-7260 apresentou certa complexidade para ser configurada pois n�o havia um t�cnico dispon�vel para auxiliar a equipe, a qual teve que aprender a trabalhar com esse hardware. Os testes de integra��o da C8051F340DK e da TS-7260, operacionalizando o rob�, exigiram processos de depura��o integrados, nos quais os problemas foram isolados e corrigidos repetidas vezes. A implementa��o dos algoritmos at� a vers�o final foi realizada paralelamente aos testes b�sicos e avan�ados, nas quais os problemas de navega��o foram detectados, isolados e corrigidos. A metodologia de testes escolhida foi elaborada pela equipe e foram efetuados v�rios experimentos com registro em v�deo at� que se atingessem os resultados finais. Tendo com base os v�deos gravados e a experi�ncia em campo observada, a equipe precisou analisar os resultados, discut�-los e extrair conclus�es para finalizar o projeto, sendo que essa tarefa representou um trabalho cient�fico.



%---------- Primeiro Capitulo ----------
\chapter{Introdu��o}

O problema do controle de navega��o de rob�s m�veis aut�nomos �
um campo da Engenharia da Computa��o que representa um grande
desafio, devido ao fato de o ambiente ser din�mico, haver sensoriamento
sujeito a ru�dos e exig�ncias de controle e tomada de decis�o em tempo
real. Um sistema de navega��o deve garantir que o rob� m�vel atinja
satisfatoriamente o destino de sua trajet�ria, enviando ao rob� comandos
necess�rios para locomo��o, de maneira precisa e suave, ao mesmo
tempo em que permite rea��es r�pidas �s mudan�as de ambiente para
evitar colis�es \cite{FRACASSO}.

Na rob�tica m�vel, existem dois principais paradigmas que guiam os
projetos de diversas arquiteturas de sistemas de navega��o: o reativo e o
deliberativo. O paradigma reativo procura reproduzir a rea��o imediata dos
animais aos est�mulos do ambiente. Geralmente, arquiteturas reativas s�o
empregadas como uma camada de n�vel inferior na navega��o de rob�s
m�veis, pois apresentam a vantagem de resposta em tempo real uma vez
que mapeiam a leitura dos sensores, diretamente, em a��es. Arquiteturas
deliberativas, por outro lado, intercalam o processo da tomada de decis�o,
desde a percep��o at� a a��o, com uma etapa de planejamento a qual
demanda grande tempo computacional, impedindo a atua��o do rob� em
tempo real. Atualmente, s�o definidas arquiteturas h�bridas, conjugando
ambos os paradigmas \cite{FRACASSO}.

Sensores de dist�ncia s�o comumente utilizados na constru��o de rob�s m�veis aut�nomos. 
Esses sensores s�o capazes de medir a dist�ncia do rob� em rela��o a um obst�culo e funcionam 
atrav�s de princ�pios f�sicos diversos, como ondas de ultra som e raios infravermelhos. A medi��o
efetuada por esses sensores � frequentemente utilizada como a entrada dos sistemas de navega��o e
� comum que esses sensores, em vers�es de tamanho reduzido, sejam encontrados em rob�s m�veis de
pequeno porte. Basicamente, um rob� m�vel aut�nomo � formado por um sistema de locomo��o, por
exemplo, um chassi com duas rodas tracionadas e uma terceira roda guia, um sistema de pot�ncia
capaz de fornecer corrente aos motores das rodas de tra��o, alimenta��o por bateria, sensores de
dist�ncia, encoders para aux�lio � odometria, um sistema microcontrolado capaz de processar os
sinais dos sensores e encoders, controlar a pot�ncia dos motores e comunicar-se atrav�s de um meio
de transmiss�o para programa��o do rob�. 

Existem metodologias de intelig�ncia artificial que podem ser empregadas em rob�tica m�vel, como a
l�gica fuzzy, redes neurais ou uma combina��o de ambas denominada ``neuro-fuzzy". O presente
trabalho � um estudo comparativo entre dois algoritmos de navega��o, que ser�o implementados para
desempenhar o comportamento reativo de um rob� m�vel aut�nomo, o qual dever� desviar obst�culos.
Um dos algoritmos � uma implementa��o utilizando l�gica fuzzy, com tomada de decis�o baseada
diretamente em regras fuzzy, enquanto que o outro � um algoritmo baseado em FCM (\emph{Fuzzy Cognitive
Maps}), que � considerado por alguns autores como um sistema neuro-fuzzy \cite{MENDONCA}. Para
realizar a compara��o desses algoritmos, ser� necess�ria uma plataforma rob�tica real, obtida pela 
reconstru��o do rob� ``Bellator", cordialmente cedido � equipe pelo professor doutor Heitor Silv�rio 
Lopes, do departamento de P�s-Gradua��o em Engenharia El�trica e Inform�tica Industrial da 
Universidade Tecnol�gica Federal do Paran� (CPGEI/UTFPR).

O Bellator � um rob� m�vel que estava em desenvolvimento em um trabalho mas foi abandonado. Este
rob� ter� de ser reconstru�do e adaptado �s necessidades da equipe, a qual levar� em conta a 
possibilidade de utiliz�-lo em projetos futuros. O Bellator ser� equipado com sensores de dist�ncia
em quantidade suficiente para alimentar a entrada do sistema de navega��o, encoders para aux�lio �
odometria, baterias para fornecimento adequado de energia, hardware e software adequados para o
controle do rob�. Os algoritmos ser�o ``embarcados'' no mesmo, evitando o uso de um PC convencional
para control�-lo ou comunica��o sem fio. Desse modo o rob� ser� aut�nomo, sendo capaz de se 
locomover e tomar decis�es sem necessidade de comunica��o com um elemento externo.

A l�gica fuzzy � uma abordagem estudada no s�timo per�odo na disciplina de sistemas inteligentes
e foi escolhida pelo conhecimento pr�vio da equipe sobre o assunto. Este trabalho se trata da
implementa��o de um controlador fuzzy para resolver o problema do controle de navegal�ao de um rob�
m�vel real. A possibilidade de trabalhar com o algoritmo FCM surgiu atrav�s do professor doutor
M�rcio Mendon�a, o qual defendeu recentemente sua tese de doutorado tratando sobre o assunto
\cite{MENDONCA}. A proposta desse algoritmo � inovadora e cujos conceitos podem ser aplicados na 
rob�tica m�vel e controladores industriais. O presente trabalho tamb�m colocar� estes conceitos em 
pr�tica com a implementa��o de um algoritmo FCM capaz de controlar o comportamento reativo do rob�. 

Atrav�s deste trabalho, a equipe espera integrar v�rios assuntos condizentes com a �rea da 
Engenharia de Computa��o. A reconstru��o do rob� levar� em conta conhecimentos de hardware e 
eletr�nica, programa��o de baixo n�vel, controle, experimentos e testes em bancada de laborat�rio. 
A programa��o dos algoritmos avaliar� a capacidade de os integrantes colocarem em pr�tica conceitos
previamente conhecidos, como a l�gica fuzzy e assimilarem novos conceitos, como o FCM. O resultado 
final ser� obtido ao se juntarem essas partes, integrando o hardware ao software de controle, que 
representar� um grande desafio � equipe e uma experi�ncia enriquecedora � n�vel de trabalho de 
gradua��o.

\section{Motiva��o}

Ao realizar uma pesquisa para an�lise do estado da arte, percebeu-se a
exist�ncia de trabalhos que apresentam novos m�todos para navega��o
aut�noma atrav�s do uso de l�gica fuzzy, entretanto,
s�o poucos os trabalhos propondo a compara��o entre os m�todos j� existentes.
Com o intuito de enriquecer essa �rea de pesquisa, a equipe optou por
desenvolver este projeto, que representa um ramo de elevado valor acad�mico.

Para realizar uma compara��o pr�tica da efici�ncia destes algoritmos, levando em conta condi��es reais de ru�dos e imperfei��es do ambiente, � indispens�vel a utiliza��o de uma plataforma rob�tica real, que agrega a necessidade de conhecimentos de hardware � equipe, por�m fornece resultados mais significativos. A plataforma foi obtida atrav�s da reconstru��o e
adapta��o do ``Bellator", o qual fora um rob� m�vel usado em um projeto
da disciplina de Oficinas de Integra��o 3 \cite{BELLATOR}. Teve-se em vista o
custo elevado de plataformas rob�ticas m�veis, como por exemplo, a X80, cujo
pre�o � de 2795 d�lares \cite{X80}, ao decidir entre adquirir uma plataforma
comercial ou reconstruir e adequar o Bellator. A possibilidade de disponibilizar
a plataforma reconstru�da e documentada para trabalhos acad�micos futuros tamb�m foi um fator decisivo para essa escolha.

Os algoritmos escolhidos para implementa��o foram o Algoritmo de Navega��o Fuzzy, baseado em infer�ncias sobre conjuntos fuzzy, e o Algoritmo de Navega��o por Mapas Cognitivos Fuzzy (FCM), tendo em vista a oportunidade de analisar o comportamento de uma nova abordagem de navega��o rob�tica em rela��o � uma abordagem j� conhecida e estudada ao longo do curso de gradua��o, ambas utilizando Fuzzy de maneiras distintas.

\section{Objetivos}
\label{rec:obj}
Os objetivos do presente trabalho s�o a reconstru��o e adequa��o da plataforma rob�tica Bellator, a descri��o do hardware e software do mesmo, antes e ap�s essa tarefa, a implementa��o dos algoritmos de navega��o e a compara��o do comportamento destes atrav�s de experimentos executados em ambientes com obst�culos n�o mapeados.

% TODO deve / dever� ?
O rob� reconstru�do e adequado dever� ser equipado com sensores de dist�ncia e encoders nas rodas, deve ser alimentado atrav�s de baterias que forne�am adequadamente energia ao sistema microcontrolado, sensores e motores, deve ser capaz de ajustar a velocidade de cada roda de maneira independente, deve processar os sinais dos sensores de dist�ncia e dos encoders e implementar rotinas para disponibilizar esses dados.

Os algoritmos de navega��o devem ser executados em um hardware independente acoplado � plataforma, visando tanto reduzir atrasos de comunica��o quanto tornar o rob� efetivamente aut�nomo. Esse hardware dever� ser capaz de ler os dados dos sensores do rob�, como fonte de dados para os algoritmos, e enviar comandos de movimenta��o ao mesmo. Cada teste dos algoritmos deve ser realizado em um ambiente igual para ambos, sendo que o objetivo de cada algoritmo � guiar o rob� no deslocamento, desviando-o de poss�veis obst�culos, tendo como entradas os valores de leitura dos sensores de dist�ncia e dos encoders e como sa�da, a velocidade e dire��o do rob�.

\section{Metodologia}
A metodologia deste trabalho foi dividida em quatro etapas: reconstru��o e adequa��o do rob� Bellator, estudo, implementa��o e testes dos algoritmos de navega��o. A reconstru��o e adequa��o consistiram na avalia��o do estado do rob�, o projeto e implementa��o do hardware e software necess�rios para o funcionamento adequado do mesmo. Durante a avalia��o, foram testados os sensores de dist�ncia, os encoders das rodas, baterias, motores e as pontes H. A elabora��o do hardware consistiu na confec��o de uma placa de roteamento para alimentar os sensores e encoders, tratar os sinais dos enconders amplificando-os, rotear os sinais dos sensores de dist�ncia e encoders aos respectivos pinos de entrada do microcontrolador e rotear os sinais de PWM gerados por este �s respectivas pontes H do rob�. O software consistiu na adequa��o do firmware previamente dispobilizado no projeto Bellator \cite{BELLATOR}, nas quais as rotinas de leitura dos sensores, comunica��o de dados, recep��o de comandos de controle do rob� e protocolo de comunica��o foram atualizadas de acordo com as necessidades desse projeto. O hardware acoplado � plataforma foi configurado para comunicar-se com o microcontrolador e executar os algoritmos de navega��o.

O estudo dos algoritmos de navega��o consistiu na revis�o bibliogr�fica da L�gica Fuzzy e do FCM, a implementa��o consistiu no desenvolvimento dos c�digos em linguagem C++, compilados para execu��o no hardware acoplado, e os testes consistiram na realiza��o de diversos experimentos nos quais o rob� teve que navegar guiando-se exclusivamente pelos algoritmos implementados.

\section{Apresenta��o do Documento}

% TODO rever isso (seis cap�tulos ? s� tem cinco listados)
Esta monografia est� dividida em cinco cap�tulos: o primeiro corresponde � introdu��o, na qual s�o apresentadas a motiva��o, os objetivos e a metodologia empregada. O segundo cap�tulo � a fundamenta��o te�rica do projeto, em que s�o apresentados o estado inicial e o estado final do rob� ap�s a reconstru��o, a explica��o da L�gica Fuzzy e da abordagem FCM. O terceiro cap�tulo � o desenvolvimento do trabalho, no qual s�o descritas as atividades realizadas pela equipe, incluindo a reconstru��o e adequa��o do rob�, o projeto e a implementa��o dos algoritmos, o desenvolvimento dos testes em campo e an�lise dos resultados. O quarto cap�tulo � a conclus�o do projeto e o quinto lista as refer�cias bibliogr�ficas, seguidas dos anexos e ap�ndices.


\chapter{Fundamenta��o Te�rica}
\label{chap:fundteor}

Este cap�tulo apresentar� a fundamenta��o te�rica desse trabalho de conclus�o de curso. Ser� descrito o estado inicial do projeto e do rob� Bellator, com uma vis�o geral deste ao ser entregue � equipe, a especifica��o do rob� ap�s a reconstru��o, a apresenta��o da L�gica Fuzzy e da metodologia de Mapas Cognitivos Fuzzy (FCM).

%---------- Estado inicial do Projeto ----------
\section{Estado Inicial do Projeto} \label{sec:estpro}

Esta se��o visa descrever com quais recursos a equipe iniciou a execu��o do trabalho, tanto materiais como intelectuais, ou seja, a situa��o do rob� e seus componentes de hardware, o principal e mais importante recurso material desse projeto, os componentes de software e a documenta��o de ambos, da forma como foi entregue � equipe.

\subsection{Vis�o Geral}

O rob� disponibilizado � equipe para a realiza��o desse trabalho, o Bellator, � resultado de um projeto n�o conclu�do, visando a implementa��o de uma plataforma rob�tica controlada remotamente por joystick. Para tal, o rob� foi dividido em tr�s camadas: baixo n�vel, alto n�vel e supervis�rio. A camada de baixo n�vel seria respons�vel por controlar os motores do rob�, bem como receber leituras dos sensores do mesmo. A camada de alto n�vel comunicar-se-ia com a camada de baixo n�vel via conex�o serial, faria obten��o de v�deo atrav�s de uma webcam e comunicar-se-ia com a camada supervis�ria atrav�s de uma conex�o sem fio. Esta �ltima seria a interface com o usu�rio para realizar o controle remoto do rob�. O diagrama esquem�tico da figura \ref{fig:diagsis} demonstra essa situa��o.

\begin{figure}[!htb]
	\centering
	\includegraphics[width=0.8\textwidth]{./figs/diagsis.png}
	\caption[Diagrama original do Bellator]{Diagrama original do projeto Bellator, retirado da monografia do mesmo.}
	\fonte{\cite{BELLATOR}}
	\label{fig:diagsis}
\end{figure}

A partir da figura \ref{fig:diagsis}, o funcionamento do projeto Bellator pode ser explicado. A camada de baixo n�vel � composta pelo rob� Bellator, equipado com dois motores el�tricos Bosch FPG 12V, cinco sensores de dist�ncia ``2Y0A02F98"~ da Sharp, uma bateria Unybatt 12V-7,2 Amp�re hora, duas pontes H e a placa microcontrolada C8051F340, capaz de ler e converter leituras de tens�o anal�gicas dos sensores do rob� bem como produzir sinais de controle para os motores do rob�. Esta placa est� conectada � camada de alto n�vel, composta por um PC Embarcado VIA EPIA ME60000 Mini-ITX com sistema operacional Linux, atrav�s de uma conex�o serial. Utilizando-se de um protocolo de comunica��o, esse PC embarcado envia comandos de movimenta��o para a camada de baixo n�vel e recebe as leituras dos sensores obtidas pela camada de baixo n�vel. O PC embarcado tamb�m comunica-se com a camada de alto n�vel para receber comandos de movimenta��o do usu�rio e enviar as leituras dos sensores para o mesmo. Al�m disso, o PC embarcado envia, tamb�m via comunica��o sem fio, um stream de v�deo gerado por uma webcam Genius iLook 316. Finalmente, o software supervisor remoto, executado em um PC com m�quina virtual Java, fornece as informa��es recebidas da camada de alto n�vel para o usu�rio, permitindo-o tomar decis�es sobre a locomo��o do rob�. O software tamb�m recebe comandos de movimenta��o do usu�rio, gerados em um Joystick do videogame Sony Playstation 2, enviando-os para a camada de alto n�vel pela mesma conex�o.

O leitor pode observar que o projeto Bellator apresenta objetivos muito distintos do projeto ``Algoritmos de Navega��o Fuzzy: Uma An�lise Qualitativa". Assim, n�o ser� feita uma descri��o detalhada de todos os componentes do projeto Bellator, entretanto, pode-se consultar a refer�ncia \cite{BELLATOR} para mais informa��es. A seguir, ser� descrito como o rob� foi recebido pela equipe e quais componentes foram reaproveitados.

\subsection{Recebimento do Rob�}
\label{sec:recrobo}

O rob� Bellator foi entregue � equipe em Abril de 2011, em uma caixa, desmontado, juntamente com toda a documenta��o dispon�vel por m�dia digital. A caixa continha os seguintes itens:

\begin{itemize}
\item Chassi do rob� Bellator com dois motores Bosch FPG12V e pontes H acoplados;
\item Cinco sensores de dist�ncia ``2Y0A02F98"~ da Sharp;
\item Duas baterias Unybatt 12V-7,2 Amp�re hora;
\item Uma placa micro-controlada C8051F340;
\item Uma placa de roteamento, produzida pelo projeto Bellator;
\item Um PC Embarcado VIA EPIA ME60000 Mini-ITX.
\end{itemize}

O chassi do rob� e seus componentes acoplados s�o a base da plataforma rob�tica a ser utilizada pela equipe e s�o cr�ticas para a execu��o desse trabalho. Os sensores de dist�ncia, um a menos que os dispon�veis no projeto Bellator, s�o essenciais para a localiza��o do rob� e de obst�culos. A placa microcontrolada tamb�m � um recurso cr�tico, pois � o componente respons�vel por todo o controle de baixo n�vel do rob� cujo software e a documenta��o ser�o reaproveitados do projeto Bellator. As baterias s�o um recurso necess�rio e de f�cil aquisi��o, ao contr�rio dos outros componentes citados.

Alguns itens mencionados na se��o anterior, referentes ao projeto Bellator, n�o foram recebidos ou n�o ser�o utilizados nesse trabalho. A webcam e joystick n�o foram entregues pois n�o ser�o necess�rios, visto que um sistema de navega��o aut�nomo n�o necessita de joystick e esse trabalho n�o abordar� navega��o atrav�s de imagem de v�deo. A placa de roteamento entregue ser� utilizada nos testes dos componentes, visto que � essencial para o funcionamento do rob�. Esta placa ser� reprojetada e reconstru�da. O PC embarcado foi entregue destitu�do de qualquer documenta��o. Al�m disso, como o novo objetivo do rob� n�o necessita de comunica��o sem fio e n�o utilizar� stream de v�deo, motivo principal para a utiliza��o deste PC no projeto anterior, a equipe optou por descartar este recurso do projeto e utilizar outra placa mais simples e menor, que ser� descrita em detalhes na se��o \ref{chap:esprob}.

A documenta��o dispon�vel � equipe, produzida durante o projeto Bellator, descreve em detalhes os componentes de hardware dessa plataforma rob�tica, o software de controle supervis�rio e a camada de baixo n�vel, ou seja, o software da placa C8051F340\cite{BELLATOR}. Essa documenta��o ser� utilizada como refer�ncia para o reaproveitamento do projeto Bellator nesse trabalho, com exce��o da documenta��o referente ao software de controle supervis�rio, que n�o ser� utilizado. O processo de reconstru��o e adapta��o da plataforma rob�tica � descrito em detalhes no cap�tulo \ref{chap:desenv}.

\subsection{Considera��es}
A possibilidade de reaproveitamento parcial do projeto Bellator e o recebimento desse material consistiram uma importante etapa nesse trabalho. A plataforma Bellator � uma op��o de recurso ao apoio do estudo qualitativo proposto nesse TCC.

%---------- Especifica��es do hardware do rob� ----------
\section{Especifica��es do Rob�}
\label{chap:esprob}

Esta se��o visa descrever as especifica��es do hardware do rob� Bellator ap�s a reconstru��o do mesmo.

\subsection{Sensor IR 2Y0A02F98}
\label{sec:sensores}
Com o objetivo de auxiliar a navega��o do rob� e fazer varreduras do ambiente, o rob� possui cinco sensores anal�gicos de dist�ncia, por infravermelho, modelo 2Y0A02F98 da Sharp, dispostos uniformemente nas laterais da plataforma Bellator. O c�digo dos sensores para convers�o e envio pela serial j� estava pronto, como resultado do projeto Bellator. A documenta��o relacionada a esse sensor foi retirada do relat�rio de tal projeto \cite{BELLATOR}, adaptada � nova situa��o. Esse sensor mede dist�ncias no intervalo de 20 a 150 cent�metros \cite{datasheetsensor}, sendo que os valores de tens�o de resposta do sensor seguem a curva mostrada na figura \ref{curva}.

\begin{figure}[!htb]
  \centering
  \includegraphics[width=0.5\textwidth]{./figs/curvaresp.png}
  \caption[Curva de resposta do sensor de dist�ncia]{Curva de resposta do sensor de dist�ncia.}
  \fonte{\cite{datasheetsensor}}
  \label{curva}
\end{figure}

Pode-se observar que o modelo � pouco influenciado pelas cores dos objetos refletidos, isto devido ao m�todo de medi��o baseado em triangula��o \cite{datasheetsensor}. O sensor possui uma tens�o de alimenta��o recomendada na faixa de 4,5 a 5,5 V, a qual n�o � atendida pela plataforma utilizada \cite{datasheetkit}, sendo necess�ria utiliza��o de alimenta��o especial. Essa alimenta��o � realizada atrav�s de um circuito regulador de tens�o, o qual utiliza alimenta��o da pr�pria bateria acoplada ao rob�. O c�lculo dos valores dos resistores foram baseados na equa��o \ref{eqRegulador}.

\begin{equation}
\label{eqRegulador}
V_{OUT} = 1,25V (1 + R_{2}/R_{1})
\end{equation}

O diagrama esquem�tico do regulador de tens�o � mostrado na figura \ref{regulador}.

\begin{figure}[!htb]
  \centering
  \includegraphics[width=0.5\textwidth]{./figs/regulador.jpg}
  \caption[Diagrama esquem�tico do regulador de tens�o dos sensores de dist�ncia]{Diagrama esquem�tico do regulador de tens�o dos sensores de dist�ncia.}
  \fonte{\cite{BELLATOR}}
  \label{regulador}
\end{figure}

\begin{figure}[!htb]
  \centering
  \includegraphics[width=0.8\textwidth]{./figs/dimsens.png}
  \caption[Dimens�es do sensor GP2Y0A02F98YK]{Dimens�es do sensor GP2Y0A02F98YK.}
  \fonte{\cite{datasheetsensor}}
  \label{dimsens}
\end{figure}

As respectivas dimens�es do sensor IR, em mil�metros, s�o mostradas na figura \ref{dimsens}. O modelo em quest�o � adequado ao projeto pois, como sua principal finalidade � a de auxiliar a navega��o do rob� em ambientes fechados, sua faixa de resposta de 20 a 150 cent�metros � suficiente para detec��o de objetos. Contudo, h� a possibilidade de em um projeto futuro serem acrescentados outros tipos de sensores mais precisos voltados � medi��o de dist�ncias menores.

\subsection{C8051F340DK}
\label{sec:C8051F340DK}

O C8051F340 � uma unidade microcontroladora (MCU) equipada com um processador  da fam�lia 8051 e v�rios dispositivos perif�ricos dispostos em uma placa de circuito impresso. As especifica��es dessa unidade foram retiradas do relat�rio do projeto Bellator \cite{BELLATOR}. A C8051F340 possui as seguintes caracter�sticas \cite{datasheetkit}:

\begin{itemize}
  \item Conversor ADC 10 bits de at� 200 ksps (amostras por segundo);
  \item Dois comparadores;
  \item Brown-out Reset e Power-on Reset;
  \item Tens�o de Refer�ncia interna;
  \item Porta USB 2.0;
  \item Duas interfaces seriais (UART) e uma interface SPI;
  \item Fonte de Alimenta��o de 2.7 at� 5.25V regulada internamente;
  \item Micro-processador 8051 de at� 48 MIPS;
  \item 4352 Bytes de mem�ria RAM;
  \item 40 Portas I/O;
  \item 4 Timers de 16 bits;
  \item Sele��o de Clock interno de alta ou baixa velocidade ou clock externo.
\end{itemize}

O diagrama em blocos do kit, retirado do datasheet � apresentado na figura \ref{dbloc}.

\begin{figure}[!htb]
  \centering
  \includegraphics[width=0.8\textwidth]{./figs/dbloc.png}
  \caption[Diagrama em blocos do kit C8051F340DK]{Diagrama em blocos do kit C8051F340DK.}
  \fonte{\cite{datasheetkit}}
  \label{dbloc}
\end{figure}

\subsection{Placa TS-7260}
\label{sec:espts7260}

A placa TS-7260 � um sistema embarcado equipado com um processador ARM e sistema operacional Linux. O sistema possui perif�ricos para realiza��o de comunica��o serial, ethernet, usb, entre outros. A lista a seguir descreve os componentes mais importantes para o projeto. Os dados, bem como a figura \ref{fotots}, foram retirados do datasheet: \cite{TS-7260}.

\begin{itemize}
  \item Processador ARM9 de 200MHz baseado no Cirrus EP9302
  \item 32MB de mem�ria NAND Flash
  \item 32MB de mem�ria SDRAM
  \item Consumo menor que 1 Watt mesmo em capacidade m�xima
  \item Porta Ethernet 10/100
  \item Duas portas USB 2.0
  \item Entrada de 4.5 a 20 Volts
  \item Dimens�es: 9.7 cm por 11.5cm
\end{itemize}

\begin{figure}[!htb]
  \centering
  \includegraphics[width=0.8\textwidth]{./figs/fotots.png}
  \caption[Foto esquem�tica da placa TS-7260]{Foto esquem�tica da placa TS-7260.}
  \fonte{\cite{tsmanual}}
  \label{fotots}
\end{figure}

Mais informa��es sobre o sistema, bem como a documenta��o completa, est�o dispon�veis no manual da placa, dispon�vel nas refer�ncias bibliogr�ficas: \cite{tsmanual}.

\subsection{Rob� Bellator}
\label{sec:robobellator} %TODO Atualizar figura/Dados

O rob� Bellator, como mencionado na se��o \ref{sec:estpro}, foi reconstru�do e adaptado ao projeto. O rob� possui duas rodas de tra��o e uma roda guia. As rodas de tra��o est�o nas laterais da parte dianteira do rob� e possuem di�metro de 20 cent�metros e espessura de aproximadamente 4 cent�metros. Ambas possuem um encoder de quadratura acoplado, o qual tem resolu��o de 1800 pulsos por volta. Cada encoder est� acoplado a um circuito �ptico HEDS-9700 que gera uma onda quadrada na sa�da � medida que o encoder gira, conforme � ilustrado na figura \ref{fig:HEDS-9700}. A roda guia est� no centro da parte traseira do rob� e possui di�metro de aproximadamente 6 cent�metros e espessura de 2 cent�metros. Todas as rodas s�o da marca Schioppa. Os outros componentes do rob� est�o listados a seguir:
\begin{itemize}
  \item[-] 2 Motores Bosch FPG 12V;
  \item[-] Bateria Unybatt 12V-7,2 Amp�re-hora;
  \item[-] Duas pontes H L 298.
\end{itemize}

\begin{figure}[!htb]
  \centering
  \includegraphics[width=0.6\textwidth]{./figs/HEDS-9700.png}
  \caption[Formas de onda de sa�da do encoder �ptico]
  {Formas de onda de sa�da do encoder �ptico.}
  \label{fig:HEDS-9700}
  \fonte{\cite{HEDS9700}}
\end{figure}

Uma imagem do rob�, parcialmente montado, pode ser visualizada na figura \ref{bellator}.

\begin{figure}[!htb]
  \centering
  \includegraphics[width=0.8\textwidth]{./figs/bellator.jpg}
  \caption[Rob� Bellator parcialmente montado]{Rob� Bellator parcialmente montado.}
  \fonte{Autoria pr�pria}
  \label{bellator}
\end{figure}

\subsection{Placa de Roteamento}
\label{sec:espplacaroteamento}

A placa de roteamento � um componente desenvolvido com base na placa dispon�vel do projeto Bellator. Como explicado na se��o \ref{sec:recrobo}, a placa de roteamento recebida foi reprojetada e reconstru�da pela equipe. As especifica��es da nova placa de roteamento s�o:
\begin{itemize}
    \item Dimens�es: 5 x 9 cm;
    \item Trilhas de cobre de aproximadamente 1 mm;
    \item Regulador de tens�o LM317T: Entrada at� 40V, Sa�da 5,05V;
    \item Buffer para PWM: 74LS244;
    \item Conectores para cabos flat, PWM, Sensores e Encoders.
\end{itemize}

O diagrama esquem�tico da placa descrita pode ser observado na figura \ref{roteamentosch}.

\begin{figure}[!htb]
  \centering
  \includegraphics[width=0.8\textwidth]{./figs/roteamento_sch.png}
  \caption[Diagrama esquem�tico da placa de roteamento]
  {Diagrama esquem�tico da placa de roteamento.}
  \fonte{Autoria pr�pria}
  \label{roteamentosch}
\end{figure}

O projeto da placa, m�todos de desenvolvimento, requisitos, entre outros foram descritos de forma detalhada na se��o \ref{sec:desroteamento}.

\subsection{Considera��es}
O rob� Bellator foi indicado para esse projeto pois apresenta as capacidades de sensoriamento e movimenta��o adequadas para a plataforma rob�tica aut�noma necess�ria para o estudo qualitativo dos algoritmos de navega��o. 


\section{L�gica Fuzzy}
\label{sec:logfuzzy}
Esta se��o descreve os conceitos fundamentais utilizados pela equipe para o entendimento e implementa��o do algoritmo de navega��o fuzzy, descrito em detalhes na se��o \ref{sec:algfuzzy}.

\subsection{Conjuntos Fuzzy}

A teoria de conjuntos fuzzy foi elaborada inicialmente por Lofti Zadeh \cite{ZADEH}, visando explorar a possibilidade de criar um novo crit�rio de afilia��o � conjuntos. Na teoria cl�ssica de conjuntos, um elemento pode apenas pertencer ou n�o a um conjunto, sendo imposs�vel um n�vel de pertin�ncia parcial. J� em um conjunto fuzzy, isto torna-se poss�vel.
Um conjunto fuzzy pode ser definido por um conjunto de pares ordenados com o elemento e sua pertin�ncia ao conjunto fuzzy. Seja F um conjunto fuzzy e X um conjunto de objetos arbitr�rios, tem-se:

\begin{equation}
F=\{(x,f(x)),x\in X\}, f(x) = [0,1]
\end{equation}

Assim sendo, considere um conjunto ``A"~ simples que contenha tudo o que tem sabor doce. Neste conjunto uma barra de chocolate � doce da mesma forma que cana de a�ucar, visto que ambos pertencem ao conjunto, ou seja:
(barra de chocolate) $\in$ A e (cana de a�ucar) $\in$ A
Em um conjunto fuzzy B que contemple tudo o que tem sabor doce, torna-se poss�vel atribuir um n�vel de afilia��o ao conjunto atrav�s de uma fun��o de pertin�ncia f  permitindo dizer que, por exemplo, a cana de a�ucar � doce com n�vel de pertinencia 1, enquanto que a barra de chocolate � doce com n�vel de pertin�ncia 0.8, ou seja:\\*

\begin{center}
((cana de a�ucar),f(cana de a�ucar)) $\in$ B, f(cana de a�ucar) = 1\\*
((barra de chocolate),f(barra de chocolate)) $\in$ B, f(barra de chocolate) = 0.8\\*
\end{center}

Isto aproxima-se mais da forma como a cogni��o e intui��o humana funcionam, que frequentemente usa-se de palavras como ``mais"~, ``muito"~, ``pouco"~, entre outras, para definir graus de pertin�ncia a conjuntos de uma forma subjetiva.

\subsection{Vari�vel Lingu�stica}
\label{varling}

Uma aplica��o direta de conjuntos fuzzy � a defini��o de vari�veis lingu�sticas. \cite{PEDRYCZ}
Considerando que vari�vel x pode assumir um valor qualquer dentro de um dado conjunto A, pode-se definir como uma vari�vel lingu�stica como uma vari�vel cujo conjunto A de valores poss�veis para a mesma � um conjunto de termos lingu�sticos, tais como, por exemplo, alto, baixo, curto, longo, entre outros. Pode-se estender este conceito associando cada termo lingu�stico poss�vel de uma vari�vel lingu�stica a um conjunto fuzzy.

Para entender esta defini��o, considere a vari�vel lingu�stica Temperatura (T) composta pelos termos lingu�sticos frio, morno e quente:
\begin{equation}
T = \{frio, morno, quente\}
\end{equation}
Considere tamb�m os seguintes conjuntos fuzzy:
\begin{equation}
	\begin{array}{lcl}
		F & = & \{(t,f(t)),t \in \mathbb{R}\} \\
		M & = & \{(t,m(t)),t \in \mathbb{R}\} \\
		Q & = & \{(t,q(t)),t \in \mathbb{R}\} \\
	\end{array}
\end{equation}
sendo \emph{\mbox{f(t), m(t) e q(t)}} fun��es de pertin�ncia, respectivamente, aos conjuntos fuzzy \emph{\mbox{F, M e Q}}, com valores de pertin�ncia pertencentes ao intervalo [0,1] e \emph{t} uma vari�vel representando a temperatura em um material qualquer. Considere ainda a associa��o dos conjuntos fuzzy \emph{\mbox{F, M e Q}} aos termos lingu�sticos frio, morno e quente, respectivamente. Neste cen�rio, um dado valor para a vari�vel \emph{t} pode ser traduzido em um valor equivalente para a vari�vel lingu�stica \emph{T}, dependendo apenas da defini��o das fun��es de pertin�ncia \emph{\mbox{f(t), m(t) e q(t)}}.

\begin{figure}[!htb]
	\centering
	\includegraphics[width=0.8\textwidth]{./figs/fuzzygraph.png}
	\caption[Exemplo de Pertin�ncias Fuzzy]{Fun��es de Pertin�ncia f(t), m(t), q(t).}
	\label{fig:fuzzygraph}
\end{figure}

Se considerarmos, por exemplo, a defini��o gr�fica para as fun��es \emph{\mbox{f(t), m(t) e q(t)}} dada na figura \ref{fig:fuzzygraph}, e tr�s valores de temperatura, \emph{\mbox{t1 = 30, t2 = 50 e t3 = 90}}, temos que os valores correspondentes de temperatura para a vari�vel lingu�stica T s�o \mbox{T1 = (0.5 frio, 0.5 morno, 0.0 quente)}, \mbox{T2 = (0.0 frio, 1.0 morno, 0.0 quente)} e \mbox{T3 = (0.0 frio, 0.0 morno, 1.0 quente)}, respectivamente. Ou seja, informalmente, a temperatura t1 � representa que o material est� ``meio frio"~ e ``meio morno"~, a temperatura t2 indica que est� ``morno"~ e a temperatura t3, por sua vez, ``quente"~. Este processo de convers�o para uma vari�vel lingu�stica � comumente chamado de ``fuzzifica��o".


\subsection{Controle Fuzzy}
\label{sec:fuzzycontrol}
Ap�s definidas as vari�veis lingu�sticas, conjuntos fuzzy e suas fun��es de pertin�ncia, descritas na se��o \ref{varling}, pode-se construir um controle fuzzy baseado em um conjunto de regras de infer�ncia.

Regras de infer�ncia sobre conjuntos fuzzy podem ser categorizadas como uma generaliza��o do \emph{modus ponens} bin�rio. Em l�gica bin�ria, dada regra ``se X ent�o Y", onde X e Y s�o vari�veis bin�rias, a partir do momento que a premissa, representada pela vari�vel X, assume valor l�gico verdadeiro, a conclus�o, dada por Y, � verdadeira tamb�m. Em l�gica fuzzy, o mesmo racioc�nio � valido, por�m X e Y s�o vari�veis lingu�sticas, e as regras de infer�ncia s�o definidas a partir dos valores que estas vari�veis lingu�sticas podem assumir, permitindo inclusive ativa��o parcial de regras de infer�ncia. Extendendo o exemplo das temperaturas, apresentado na se��o \ref{varling}, imagine que deseja-se controlar a velocidade de um \emph{cooler} de processador, de acordo com a temperatura que este se encontra, utilizando um controle fuzzy. As regras de infer�ncia para este controle podem ser, por exemplo:
\begin{center}
    Se \emph{morno} ent�o \emph{medio}\\*
    Se \emph{quente} ent�o \emph{forte}
\end{center}
Sendo ``medio"~ e ``forte"~ valores poss�veis da vari�vel lingu�stica Velocidade (V), que controla a velocidade do \emph{cooler}, com ``medio"~ correspondendo a fun��o de pertin�ncia vm e ``forte"~ correspondendo a vf, apresentados na figura \ref{fig:defuzzygraph}:

\begin{figure}[!htb]
	\centering
	\includegraphics[width=0.8\textwidth]{./figs/defuzzygraph.png}
	\caption[Exemplo de Pertin�ncias Fuzzy - Defuzzifica��o]{Fun��es de Pertin�ncia vm e vf.}
	\label{fig:defuzzygraph}
\end{figure}

De acordo com estas regras de infer�ncia, se a temperatura \emph{fuzzificada} for inteiramente fria, nenhuma regra ser� ativada e a velocidade do \emph{cooler} ser� nula. Por�m, se a temperatura for maior que a m�nima necess�ria para come�ar a ser classificada como morna, haver� ativa��o integral ou parcial de uma ou ambas as regras de infer�ncia. Neste caso, � necess�rio determinar o grau de ativa��o de cada uma das regras e produzir uma sa�da que contemple estes graus de ativa��o, que � o processo inverso � \emph{fuzzifica��o}, a \emph{defuzzifica��o}. Um destes m�todos � a m�dia do m�ximo, que consiste da m�dia ponderada dos m�ximos de cada valor fuzzy de sa�da, com os pesos correspondendo �s ativa��es das regras de infer�ncia. Considerando novamente o exemplo do controle de velocidade de um \emph{cooler} e considerando que, em um determinado momento, a temperatura est� ``meio morno"~ e ``meio quente"~, ou seja, 0.5 de pertin�ncia � classe ``morno"~ e a classe ``quente"~, ambas as regras ser�o ativadas igualmente e a velocidade do \emph{cooler} ser�:
\begin{equation}
v = \frac{\left( 0.5*50 + 0.5*100 \right ) } {1} = 75
\end{equation}

\subsection{Considera��es}

O conceito de incerteza introduzido pela L�gica Fuzzy permite a modelagem de sistemas com problemas de decis�o cujas vari�veis s�o din�micas, reduzindo a influencia de ru�dos e possibilitando a constru��o de um sistema de infer�ncias an�logo � cogni��o humana. O problema de navega��o rob�tica altamente din�mico, pois as decis�es s�o tomadas sob a influ�ncia de v�rios sensores simultaneamente, todos pass�veis de ru�do, e a abordagem Fuzzy � uma op��o plaus�vel para trat�-lo.

\section{Mapas Cognitivos Fuzzy}
\label{sec:fcm}
Esta se��o tem como objetivo explicar o que s�o mapas cognitivos fuzzy, tamb�m conhecidos como FCM (Fuzzy Cognitive Maps), abordando o conceito, a estrutura, as propriedades e vantagens desse modelo, apresentar os passos para contru��o de um FCM e apresentar exemplos, que descrevem o uso em situa��es reais.

O modelo FCM � abordado na tese de doutorado de \cite{FCMENDONCA}. Mapas cognitivos s�o diagramas que representam liga��es entre palavras, id�ias, tarefas ou outros itens ligados a um conceito central, dispostos radialmente, intuitivamente e de acordo com a import�ncia de cada conceito. Cren�as ou afirma��es a respeito de um dom�nio de conhecimento limitado s�o expressas por palavras ou express�es lingu�sticas interligadas por rela��es de causa e efeito, que possibilitam predizer as consequ�ncias que essa organiza��o implica ao universo representado. O mapa cognitivo fuzzy � gerado quando se incluem a essa estrutura incertezas atrav�s da l�gica Fuzzy.

\begin{figure}[!htb]
    \centering
    \includegraphics[width=0.5\textwidth]{./figs/fcm.png}
    \caption[Exemplo de um FCM]{Exemplo de um FCM (grafo).}
    \fonte{\cite{FCMENDONCA}}
    \label{fig:fcm}
\end{figure}

A estrutura de um FCM � um grafo direcionado, figura \ref{fig:fcm}, em que os valores num�ricos s�o vari�veis ou conjuntos fuzzy, os ``n�s"~ s�o conceitos lingu�sticos, representados por conjuntos fuzzy e cada ``n�"~ � associado a outros n�s atrav�s de conex�es (relacionamentos), a cada qual est� associado um peso num�rico, que representa a vari�vel fuzzy relacionada ao n�vel de causalidade entre os conceitos. De acordo com \cite{MENDONCA}, um FCM suporta diversos tipos de conceitos e relacionamentos:

\begin{itemize}
    \item Conceito de n�vel: Esse conceito pode pode ser representado por um valor absoluto;
    \item Conceito de varia��o: Esse tipo de conceito representa a varia��o de um valor no tempo;
    \item Conceitos de entradas: Esses conceitos recebem um valor de entrada e podem interagir com outros conceitos;
    \item Conceitos de sa�da ou de decis�o: Esses conceitos representam o resultado das infer�ncias do FCM e n�o interagem com outros conceitos;
    \item Rela��es causais: Essas conex�es representam as rela��es de causa e efeito entre os conceitos e s�o calculadas atrav�s da matriz de pesos (matriz W);
    \item Declara��es condicionais: Esses elementos s�o as rela��es causais expressas na forma de regras \emph{se-ent�o} e s�o atualizadas temporalmente.
\end{itemize}

Na figura \ref{fig:fcm}, os conceitos (C1 a C5) podem ser atualizados atrav�s da intera��o com outros conceitos por meio das rela��es causais ($w_{i,j}$) e com seu pr�prio valor. A matriz \ref{fcm-matrix} representa o peso das rela��es causais entre os conceitos e podem ser atualizados atrav�s da equa��o \ref{fcm-update}. Esta descreve a evolu��o do FCM, na qual j � o contador das itera��es, n � o n�mero de n�s do grafo, $W_{ji}$ � o peso do arco que conecta o conceito $C_j$ ao conceito $C_i$, $A_i$ e $A_i^{anterior}$ s�o o valor do conceito $C_i$ na itera��o atual e anterior, respectivamente, e a fun��o f \ref{sigmoide} � uma fun��o do tipo sigm�ide.

\begin{equation}\label{fcm-matrix}
    w_{i,j}=\left(
       \begin{array}{ccccc}
         0 & w_{12} & 0 & 0 & w_{15} \\
         0 & 0 & w_{23} & 0 & w_{25} \\
         0 & w_{32} & 0 & w_{34} & 0 \\
         w_{41} & 0 & w_{43} & 0 & w_{45} \\
         w_{51} & 0 & 0 & w_{54} & 0 \\
       \end{array}
     \right)
\end{equation}

\begin{equation}\label{fcm-update}
A_i=f(\sum_{\substack{j=1 \\ j\neq i}}^{n} A_j \times W_{ji})+A_i^{anterior}
\end{equation}

\begin{equation}\label{sigmoide}
f(x)=\frac{1} {1+e^{-\lambda x}}
\end{equation}

Em \cite{KOSKO}, s�o apresentados os seguintes passos para constru��o de um FCM cl�ssico:

\begin{itemize}
\item Passo 1 - Identifica��o dos conceitos e das suas interconex�es ou rela��es
determinando a natureza (positiva, negativa ou neutra) das rela��es causais entre
conceitos;
\item Passo 2 - Aquisi��o de dados iniciais, atrav�s de pondera��o de opini�o de
especialistas e ou an�lise do sistema de equa��es, quando se conhece o modelo
matem�tico;
\item Passo 3 - Apresenta��o dos dados referentes � opini�o dos diversos especialistas a
um sistema l�gico fuzzy que tem como sa�da os valores dos pesos do FCM;
\item Passo 4 - Tratamento da informa��o, adapta��o e ou otimiza��o do FCM
inicialmente proposto, ajustando suas respostas �s sa�das desejadas;
\item Passo 5 - Valida��o do FCM ajustado nas condi��es de opera��o do sistema ou
processo modelado.
\end{itemize}

Um FCM apresenta as propriedades de elasticidade e estabilidade, sendo que a elasticidade, ou auto organiza��o, � a capacidade de refor�ar ou enfraquecer o peso das rela��es causais e a estabilidade � a capacidade de o mapa evoluir, estabilizando-se em um ponto fixo ou ap�s um n�mero m�ximo de itera��es. Uma vantagem do FCM � a modularidade, a qual permite que um problema complexo seja representado por v�rios mapas modulares e outra vantagem � que os pesos das rela��es causais e dos conceitos podem ser obtidos via treinamento a partir dos dados hist�ricos do sistema ou atrav�s de um algoritmo adaptativo, que atualiza os pesos constantemente.

\begin{figure}[!htb]
    \centering
    \includegraphics[width=0.5\textwidth]{./figs/fcm-exemplo-planta.png}
    \caption[Aplica��o do FCM em processo industrial]{Aplica��o do FCM em processo industrial.}
    \fonte{\cite{GROUMPOS}}
    \label{fig:fcm-exemplo-planta}
\end{figure}

Em \cite{GROUMPOS} os mapas cognitivos fuzzy s�o aplicados no controle de processos industriais. Um exemplo de aplica��o � mostrado na figura \ref{fig:fcm-exemplo-planta}, na qual � ilustrado um tanque com duas v�lvulas de entrada (V1 e V2) para diferentes tipos de l�quidos, um misturador, uma v�lvula de sa�da (V3) para o l�quido misturado e um medidor de massa espec�fica (G) que mede a quantidade de l�quido produzida. As v�lvulas V1 e V2 introduzem dois l�quidos diferentes. Durante a mistura, o medidor de massa espec�fica verifica quando o produto atingiu o ponto adequado e, desse modo, a v�lvula V3 � ativada e o produto da mistura � esvaziado. Analisando-se o problema, os seguintes conceitos podem ser definidos:

\begin{itemize}
\item Conceito 1: Volume de l�quido no tanque, o qual depende do estado das v�lvulas V1, V2 e V3;
\item Conceito 2: Estado da v�lvula 1 (fechada, aberta ou parcialmente aberta);
\item Conceito 3: Estado da v�lvula 2 (fechada, aberta ou parcialmente aberta);
\item Conceito 4: Estado da v�lvula 3 (fechada, aberta ou parcialmente aberta);
\item Conceito 5: Valor de massa espec�fica do l�quido medido pelo sensor G.
\end{itemize}

O controlador do processo deve manter as vari�veis V e G, sendo V o volume e G a massa especifica do produto no tanque, dentro das faixas de opera��o $[V_{min}, V_{max}]$ (equa��o \ref{lim-V}) e $[G_{min}, G_{max}]$ (equa��o \ref{lim-G}), respectivamente.

\begin{equation}\label{lim-V}
V_{min}<V<V_{max}
\end{equation}

\begin{equation}\label{lim-G}
G_{min}<G<G_{max}
\end{equation}

Interligando-se os conceitos atrav�s de rela��es de causa e efeito, o FCM da figura \ref{fig:fcm-exemplo-fcm} foi constru�do.

\begin{figure}[!htb]
    \centering
    \includegraphics[width=0.5\textwidth]{./figs/fcm-exemplo-fcm.png}
    \caption[Mapa Cognitivo Fuzzy]{FCM do controlador.}
    \fonte{\cite{GROUMPOS}}
    \label{fig:fcm-exemplo-fcm}
\end{figure}

 Analisando-se o conhecimento dos especialistas, os pesos das rela��es s�o dados pelas inequa��es \ref{peso-1} a \ref{peso-8}.

\begin{equation}\label{peso-1}
-0,50<w_{12}<0,30
\end{equation}
\begin{equation}\label{peso-2}
-0,40<w_{13}<0,20
\end{equation}
\begin{equation}\label{peso-3}
0,20<w_{15}<0,40
\end{equation}
\begin{equation}\label{peso-4}
0,30<w_{21}<0,40
\end{equation}
\begin{equation}\label{peso-5}
0,40>w_{31}<0,50
\end{equation}
\begin{equation}\label{peso-6}
-1,0<w_{41}<0,80
\end{equation}
\begin{equation}\label{peso-7}
0,50<w_{52}<0,70
\end{equation}
\begin{equation}\label{peso-8}
0,30<w_{54}<0,40
\end{equation}

O controlador do processo foi executado e, ap�s a estabiliza��o, obtiveram-se os pesos da matriz \ref{W-matrix} e os valores dos conceitos da matriz \ref{A-matrix}. Os limites das equa��es \ref{lim-V} e \ref{lim-G} s�o reajustados para os valores correspondentes �s equa��es \ref{lim-V-adj} e \ref{lim-G-adj}, respectivamente, correspondendo ao ponto de opera��o desejado.

\begin{equation}\label{W-matrix}
    W^{inicial}=\left(
       \begin{array}{ccccc}
         0,00 & -0,40 & -0,25 & 0,00 & 0,30 \\
         0,36 & 0,00 & 0,00 & 0,00 & w0,00 \\
         0,45 & 0,00 & 0,00 & 0,00 & 0,00 \\
         -0,90 & 0,00 & 0,00 & 0,00 & 0,00 \\
         0,00 & 0,60 & 0,00 & 0,30 & 0,00 \\
       \end{array}
     \right)
\end{equation}

\begin{equation}\label{A-matrix}
    A^{inicial}=\left(
       \begin{array}{ccccc}
         0,10 & 0,45 & 0,39 & 0,04 & 0,01 \\
       \end{array}
     \right)
\end{equation}

\begin{equation}\label{lim-V-adj}
0,68<V<0,70
\end{equation}

\begin{equation}\label{lim-G-adj}
0,78<G<0,85
\end{equation}

Nesse exemplo, a estabiliza��o (ou sintonia) foi realizada atrav�s de tr�s m�todos: RNA (Rede Neural Artificial), AG (Algoritmo Gen�tico) e PSO (Particle Swarm Optimization ou Otimiza��o por Enxame de Part�culas).

Outra aplica��o � descrita no artigo \cite{MENDONCA}, na qual a abordagem FCM � empregada em navega��o rob�tica. Nesse artigo, um modelo de FCM novo � implementado para suportar as condi��es din�micas dos sistemas de navega��o, nas quais os valores das rela��es causais s�o modificados dinamicamente atrav�s da ocorr�ncia de eventos especiais. Os autores do artigo chamaram esse modelo de ED-FCM (Event-Driven Fuzzy Cognitive Map).

\begin{figure}[!htb]
    \centering
    \includegraphics[width=0.5\textwidth]{./figs/reinforcement.png}
    \caption[Algoritmo de aprendizado por refor�o]{Algoritmo de aprendizado por refor�o.}
    \fonte{\cite{MENDONCA}}
    \label{fig:reinforcement-alg}
\end{figure}

O ajuste dos pesos das rela��es causais � efetuado por um algoritmo de aprendizado por refor�o, conforme ilustra a figura \ref{fig:reinforcement-alg}, e permite que o rob� (agente) aprenda diretamente atrav�s de sua intera��o com o ambiente. A cada instante de tempo t, o agente estabelece, por meio de seus sensores, um estado st e, de acordo com suas regras, determina uma a��o at a ser efetuada pelos atuadores. Essa a��o causa uma transi��o para o estado $s_{t+1}$ e o ambiente retorna uma medida de refor�o $r_{t+1}$, que pode ser uma recompensa (caso a a��o seja boa) ou uma puni��o (caso a a��o seja ruim).

\begin{figure}[!htb]
    \centering
    \includegraphics[width=0.5\textwidth]{./figs/fcm-robot.png}
    \caption[FCM do comportamento reativo do rob�]{FCM do comportamento reativo do rob�.}
    \fonte{\cite{MENDONCA}}
    \label{fig:fcm-robot}
\end{figure}

O FCM descreve o comportamento reativo do rob� (figura \ref{fig:fcm-robot}), no qual a leitura dos sensores de dist�ncia (esquerdo, frontal e direito) levam a uma a��o imediata que interfere no movimento. Os conceitos RS, FS e LS representam as leituras dos sensores, os conceitos LO e RO representam as decis�es de virar � esquerda ou virar � direita, respectivamente,  decis�es anteriores, representadas pelos conceitos LO(-1) e RO(-1), exercem influ�ncia sobre as decis�es atuais e a sa�da do algoritmo � representada pelos conceitos \emph{Out Left}, \emph{Out Front} e \emph{Out Right}. As rela��es causais do mapa s�o descritas na tabela \ref{tab:causal-relations} e as regras de a seguir determinam o comportamento do mapa:

\begin{enumerate}
\item SE a intensidade do sensor frontal (FS) for maior que um limiar m�dio ENT�O $W_{lim}$ aplicado para computar o relacionamento $w_3$ � o valor m�ximo de $WF_{max}$;
\item SE a intensidade do sensor frontal (FS) for menor que um limiar m�nimo ENT�O $W_{lim}$ aplicado para computar o relacionamento $w_3$ � o valor m�nimo de $WF_{min}$;
\item SE a intensidade do sensor direito (RS) for maior que um limiar m�dio ENT�O $W_{lim}$ aplicado para computar o relacionamento $w_1$ � o valor m�ximo de $WR_{max}$;
\item SE a intensidade do sensor direito (RS) for menor que um limiar m�nimo ENT�O $W_{lim}$ aplicado para computar o relacionamento $w_1$ � o valor m�nimo de $WR_{min}$;
\item SE a intensidade do sensor esquerdo (LS) for maior que um limiar m�dio ENT�O $W_{lim}$ aplicado para computar o relacionamento $w_5$ � o valor m�ximo de $WL_{max}$;
\item SE a intensidade do sensor direito (LS) for menor que um limiar m�nimo ENT�O $W_{lim}$ aplicado para computar o relacionamento $w_5$ � o valor m�nimo de $WL_{min}$.
\end{enumerate}

\begin{table}[htb!]
	\centering
	\caption[Rela��es causais do controlador do rob�]{Rela��es causais do controlador do rob�.}		
	\begin{tabular}[!htb]{ l l l l }
	  \hline
	  Rela��o causal & Descri��o & Efeito & Intensidade \\
	  \hline
	  $w_1$ & Sensor direito (RS) incluencia a sa�da esquerda (LO) & Positivo & Forte \\
	  $w_2$ & Sensor frontal (FS) incluencia a sa�da esquerda (LO) & Positivo & M�dio \\
	  $w_3$ & Sensor frontal (FS) incluencia a sa�da frontal (FO) & Positivo & Forte \\
	  $w_4$ & Sensor frontal (FS) incluencia a sa�da direita (RO) & Positivo & M�dio \\
	  $w_5$ & Sensor esquerdo (LS) incluencia a sa�da direita (RO) & Positivo & Forte \\
	  $w_6$ & Sensor esquerdo (LS) incluencia a sa�da direita (RO) & Negativo & Fraco \\
	  $w_7$ & Sensor direito (RS) incluencia a sa�da esquerda (LO) & Negativo & Fraco \\
	  $w_8$ & Sensor direito (RS) incluencia a sa�da direita (RO) & Negativo & Fraco \\
	  $w_9$ & Sensor esquerdo (LS) incluencia a sa�da esquerda (LO) & Negativo & Fraco \\
	  \hline  	
	\end{tabular}
	\fonte{\cite{MENDONCA}}
	\label{tab:causal-relations}
\end{table}

Essas regras determinam a pol�tica de mudan�a de estados do mapa e os pesos dos relacionamentos s�o respons�veis pelas decis�es de o rob� virar � esquerda, acelerar ou virar � direita. Nesse contexto, o valor atual desses pesos depende da diferen�a entre os valores anteriores e o valor m�ximo admiss�vel ponderado por um fator $\gamma$ . O incremento dos pesos tamb�m leva em conta o valor da recompensa ou punic�o (r) e de um fator de aprendizagem $\alpha$ , os quais est�o associados ao algoritmo de aprendizado por refor�o escolhido, equa��o \ref{q-learning}.

\begin{equation}\label{q-learning}
w_i(k)=w_i(k-1)+\alpha \times [r+\gamma \times W_{lim}-w_i(k-1)]
\end{equation}

Por fim, nesse artigo, s�o descritos os resultados nos quais o rob�, em simula��o, foi capaz de desviar obst�culos � direita e � esquerda do mesmo ao longo da traget�ria.

\subsection{Considera��es}
O FCM representa um problema em termos de conceitos e rela��es causais, podendo ser empregado em controladores de processos industriais ou no controle de rob�s aut�nomos. O problema do desvio de obst�culos em navega��o rob�tica p�de ser modelado, conforme o exemplo apresentado (figura \ref{fig:fcm-robot}), atrav�s de tr�s conceitos de entrada (RS, FS e LS), dois conceitos de decis�o (RO e LO), tr�s conceitos de sa�da (Right Out, Front Out e Left Out), rela��es causais, regras \emph{SE-ENT�O} e um algoritmo de aprendizado. O controlador proposto permitiu que o rob� desviasse obst�culos reagindo � leitura de sensores que medem a dist�ncia de objetos posicionados � esquerda e � direita do mesmo, concluido-se que a abordagem FCM � adequada para esse trabalho.



%---------- Segundo Capitulo ----------
\chapter{Desenvolvimento}
\label{chap:desenv}

Este cap�tulo cont�m em detalhes o trabalho realizado pela equipe, dividido nas seguintes se��es:
\begin{itemize}
    \item[-] Teste dos Componentes: esta se��o cont�m em detalhes quais e em que ordem foram realizados os
     testes de componentes do rob�.
    \item[-] C�digo do microcontrolador C8051F340: cont�m informa��es a respeito das mudan�as no c�digo do
    microcontrolador.
    \item[-] Placa de roteamento: cont�m as especifica��es da placa de roteamento, assim como os motivos
    que levarem a equipe a desenvolver esta placa, o projeto da mesma, e o resultado.
\end{itemize}

\section{Teste dos Componentes}
\label{sec:testecomp}

Ap�s o recebimento do rob�, foram realizados testes para garantir a funcionalidade dos componentes recebidos, j� que o rob�
estava com suas pe�as empilhadas numa caixa e n�o era poss�vel confiar no funcionamento adequado de nenhum dos componentes, al�m de que a falha de alguns componentes implicaria na impossibilidade de continuar o projeto ou em atrasos significativos. Estes testes tamb�m foram necess�rios para determinar de forma mais precisa o que poderia ser reaproveitado do projeto Bellator. De acordo com a documenta��o do projeto Bellator~\cite{BELLATOR}, o rob� deveria ser capaz de, se montado conforme as instru��es na mesma, funcionar como um sistema controlado remotamente. Como o objetivo deste projeto n�o envolve controlar
o rob� remotamente, foi testada apenas a camada de baixo n�vel.

O primeiro passo da etapa de testes foi verificar o funcionamento da placa C8051F340, pe�a fundamental para o desenvolvimento do projeto, que apresentou o funcionamento adequado, gerando os PWMs dos motores conforme necess�rio (visualizados no oscilosc�pio), e realizando a leitura dos sensores e convers�o A/D conforme esperado. Em seguida, foram iniciados os testes utilizando a placa de roteamento j� existente, que apresentou defeito. Ap�s alguns testes, foi constatado que a placa havia sido desconfigurada, v�rias soldas foram removidas, o circuito em si estava alterado. Ent�o, a equipe reorganizou a placa, realizou novos testes, mas n�o obteve sucesso. Foi ent�o verificado que
o regulador de tens�o n�o estava funcionando. Este foi substitu�do e a placa finalmente funcionou conforme esperado.

Com a placa de roteamento antiga funcionando, foi poss�vel realizar o teste dos motores, utilizando os PWMs gerados pelo
microcontrolador C8051F340 (entrada do buffer da placa de roteamento). Nesse teste, n�o ocorreram problemas, os motores funcionaram conforme esperado.

Das duas baterias inicialmente dispon�veis, uma n�o estava funcionando conforme a especifica��o, o que levou a equipe
a adquirir uma nova bateria 12V para reposi��o da bateria danificada.

Com todos os componentes acima citados testados, o que faltava para completar os testes da camada de baixo n�vel era apenas
o teste dos encoders. Esta etapa foi uma das mais dif�ceis, pois a equipe n�o tinha informa��o nem do modelo do encoder.
Depois de muito procurar, foi encontrado um datasheet de um encoder equivalente ao presente no rob�, datasheet este que foi fundamental para determinar como alimentar e testar o encoder. Em posse da informa��o de como usar o encoder, o teste foi realizado tanto para o encoder esquerdo como o direito. O encoder direito funcionou normalmente, por�m o esquerdo n�o. Ent�o, a equipe percebeu que a solda dos fios do encoder n�o estava boa. Ap�s refazer as soldas, o encoder esquerdo foi testado
novamente e funcionou.

Tendo realizado os testes dos componentes mais cr�ticos, o passo seguinte foi tentar utilizar o PC Embarcado VIA EPIA ME6000.
Ap�s muitas tentativas falhas e busca por informa��es sem resultados, a equipe optou por n�o utilizar este componente, j� que
sua documenta��o era escassa e o tempo perdido na tentativa de utiliz�-la j� estava acima do planejado. O PC Embarcado foi substitu�do pela placa TS-7260, que possui uma documenta��o muito melhor, poder de processamento superior, e funcionou
nos primeiros testes.

\section{C�digo do microcontrolador C8051F340}
\label{sec:codmicro}
Uma das necessidades do c�digo do microcontrolador � o tratamento dos sinais dos encoders para obter informa��es de odometria, informa��es estas essenciais para a navega��o aut�noma. O c�digo do projeto Bellator~\cite{BELLATOR} n�o continha o tratamento
dos sinais dos encoders. Portanto, foi necess�rio alterar o c�digo do microcontrolador para tratar os sinais
dos encoders e enviar informa��es de odometria. Para tal, foram utilizadas duas interrup��es externas da placa C8051F340 (uma
para cada encoder). Nestas interrup��es, cada pulso do encoder � contado (obviamente cada encoder possui seu contador separadamente). A informa��o de odometria � ent�o enviada atrav�s da comunica��o serial. Como o encoder d� 1800 pulsos por volta, a resolu��o � de 360/1800 = 0.2 graus.  Contudo, tamanha precis�o � exagerada, e a equipe optou por enviar atrav�s apenas uma mensagem (atrav�s da porta serial) a cada 180 pulsos do encoder, ou seja, 36 graus de rota��o ou um d�cimo de volta.

\section{Placa de Roteamento}
\label{sec:desroteamento}

A placa de roteamento � um componente fundamental do projeto, respons�vel por realizar a interliga��o entre a placa C8051F340 e os componentes de hardware do rob�. A equipe constatou a necessidade deste componente devido aos seguintes fatores:

\begin{itemize}
    \item[-] Necessidade de alimenta��o dedicada para alguns componentes
    \item[-] Necessidade de tratamento dos sinais dos encoders
    \item[-] Roteamento das leituras de cada sensor para o pino I/O correto da C8051F340
    \item[-] Necessidade de um buffer para o PWM
    \item[-] Melhor organiza��o do rob�
\end{itemize}

Como descrito nas especifica��es do rob� (\ref{chap:esprob}), o rob� Bellator possui, dentre outros componentes, cinco sensores, dois encoders e duas pontes H. Os cinco sensores e dois encoders necessitam de alimenta��o de aproximadamente 5 Volts para opera��o. Os encoders produzem um sinal que precisa ser amplificado antes da utiliza��o, e as duas pontes H necessitam de um sinal de PWM de baixa imped�ncia. Al�m disso, o consumo de corrente total destes componentes ultrapassa a capacidade de fornecimento de corrente da C8051F340. Medi��es durante os testes com os componentes foram realizadas, e foi constatado que cada sensor de dist�ncia utiliza cerca de 30mA de corrente, enquanto os encoders utilizam 20mA cada, totalizando quase 200mA para estes componentes.

Por fim, a dificuldade de conectar, de forma pr�tica, todos os componentes do rob� com a C8051F340 bem como a quantidade excessiva de fios resultante fez com que a equipe conclu�sse que a placa de roteamento � indispens�vel.

De acordo com os fatores descritos, a equipe construiu uma lista de requisitos para a placa de roteamento:

\begin{itemize}
    \item[-] Fornecer alimenta��o de aproximadamente 5 Volts DC
    \item[-] Capacidade de corrente suficiente %Atualizar com valores!
    \item[-] Conectores pr�ticos para interface com a C8051F340 e o resto do rob�
    \item[-] Fornecer um buffer para o sinal de PWM
    \item[-] Fornecer amplifica��o para o sinal do encoder
\end{itemize}

Ap�s a an�lise dos requisitos, a equipe iniciou o desenvolvimento de uma placa de circuito impresso para atender a todos os requisitos. A necessidade de uma alimenta��o de 5 Volts tornou essencial a utiliza��o de um regulador de tens�o, o LM317, que era o mesmo utilizado na vers�o antiga da placa, e suporta corrente de at� 1.5A, valor suficiente para alimentar os componentes da placa. O buffer para o sinal de PWM foi feito atrav�s do CI 74LS244N e a amplifica��o dos sinais dos Encoders utilizando-se um amplificador operacional LM324N. O restante da placa consiste de pinos para conectar cabos flat, para a intera��o com a C8051F340, bem como pinos para conectores do hardware do rob� Bellator. O diagrama esquem�tico de tal placa foi produzido com o aux�lio do software Eagle e o resultado pode ser visualizado a seguir.

\begin{figure}[!htb]
  \centering
  \includegraphics[width=\textwidth]{./figs/roteamento_sch.png}
  \caption[Diagrama esquem�tico da placa de roteamento.]
  {Diagrama esquem�tico da placa de roteamento.}
  \label{fig:roteamento_sch}
\end{figure}

De posse do diagrama esquem�tico apresentado na figura \ref{roteamentosch} a equipe realizou o roteamento de uma placa de circuito impresso que atenda ao mesmo diagrama esquem�tico, tamb�m utilizando o software Eagle.

\begin{figure}[!htb]
  \centering
  \includegraphics[width=\textwidth]{./figs/roteamento_brd.png}
  \caption[Roteamento final produzido pela equipe.]
  {Roteamento final produzido pela equipe.}
  \label{fig:roteamento_brd}
\end{figure}

A placa de roteamento foi confeccionada manualmente em uma placa de circuito impresso. O desenho da placa foi impresso em uma folha de transpar�ncia A4 e utilzando-se uma impressora � laser. Esse desenho foi passado da traspar�ncia para a placa com o aux�lio de um ferro de passar roupa e a placa foi corro�da usando-se o percloreto de ferro. A placa foi perfurada com broca e perfurador de placa e os componentes eletr�nicos foram soldados com ferro de solda, estanho e pasta de solda. A figura \ref{fig:confeccao} ilustra o processo de confecc��o da placa.

\begin{figure}[!htb]
  \centering
  \includegraphics[width=\textwidth]{./figs/placa-confeccao.jpg}
  \caption[Processo de confec��o da placa de roteamento.]
  {Processo de confec��o da placa de roteamento.}
  \label{fig:confeccao}
\end{figure}

%Descrever os fatores. Escrever requisitos. Descrever placa. Explicar a produ��o da placa. Por que usar o Eagle.

\section{Placa TS-7260}
%Descri��o do motivo da escolha da placa, implementa��o de c�digos auxiliares aos algoritmos (comunica��o, etc)...
Esta se��o explicar� como foi estruturado o software desenvolvido para a placa
TS-7260, cuja fun��o � realizar a interface entre o software do microcontrolador
C8051F340 e o algoritmo de navega��o (Fuzzy ou FCM). 

O c�digo que roda na TS-7260 � o que efetivamente controla o rob�. A parte principal do c�digo consiste
de um loop infinito, que executa os seguintes passos:

\begin{itemize}
	\item Envio de um SYNC para o microcontrolador: isto faz com que o microcontrolador responda
		um pacote com as leituras mais recentes dos sensores de dist�ncia, juntamente com as leituras
		dos encoders (esta comunica��o � realizada via porta serial);
	\item Leitura do pacote enviado pelo microcontrolador: recebe os dados do microcontrolador e 
		os armazena;
	\item Execu��o do algoritmo de navega��o: utiliza os dados armazenados para executar o algoritmo
		de navega��o, passando os valores de leitura dos sensores de dist�ncia. O algoritmo utiliza
		os dados para realizar a infer�ncia, retornando valores que indicam qual a��o o rob� deve tomar;
	\item Ajuste dos setpoints: nesta etapa, os dados retornados pelo algoritmo de navega��o s�o
		utilizados para o c�lculo do setpoint de cada roda (velocidade desejada), o setpoint � ent�o
		alterado de acordo;
	\item Ajuste de velocidade: esta etapa consiste do ajuste dos n�veis de PWM de cada roda, e utiliza
		para tal os valores de leitura dos encoders e o setpoint do passo anterior. 
		O objetivo do ajuste de velocidade � fazer com que cada motor fique o mais pr�ximo poss�vel
		do setpoint estabelecido previamente;
	\item Envio das a��es: os n�veis de PWM definidos pelo ajuste de velocidade s�o enviados para
		o microcontrolador C8051F340, que efetiva a mudan�a.
\end{itemize}

O diagrama em blocos a seguir representa o funcionamento do software, demonstrando
tamb�m em que etapas ocorre a comunica��o com o microcontrolador e quando o algoritmo 
de navega��o � executado.

\tikzstyle{tsb} = [rectangle, draw=black!100, fill=blue!15!green!10, 
    text width=8em, text centered, rounded corners, minimum height=1cm]
\tikzstyle{transition} = [rectangle, thick, draw=black!75, fill=black!20,
	minimum size=4mm]
\tikzstyle{line} = [draw, -latex']
\tikzstyle{8051b} = [rectangle, draw=black!100, fill=red!15, rounded corners,
	minimum height=1cm, text width=8em, text centered]
\tikzstyle{navb} = [rectangle, draw=black!100, fill=yellow!30, rounded corners,
	text width=8em, minimum height=1cm, text centered]
\tikzstyle{background} = [rectangle, fill=gray!50, inner sep=4mm, rounded corners=5mm];
    
\begin{figure}[H]
	\centering
	\begin{tikzpicture}[node distance = 1.5cm, auto]
		% N�s e conex�es do diagrama
		\node [tsb] (opentty) {Configura serial};
		\node [tsb, below of=opentty] (sync) {Envia SYNC} edge [pre] (opentty);
		\node [tsb, below of=sync] (read) {Leitura do pacote} edge [pre] (sync);
		\node [tsb, below of=read] (alg) {Executa algoritmo de navega��o} edge [pre] (read);
		\node [navb, left=2cm of alg] (nrcv) {Recebe dados} edge [pre, dashed] (alg);
		\node [navb, below of=nrcv] (ninf) {Realiza a infer�ncia} edge [pre] (nrcv);
		\node [navb, below of=ninf] (nret) {Retorna as a��es} edge [pre] (ninf);
		\node [tsb, right=2cm of nret] (sadj) {Ajuste dos setpoints} edge [pre, dashed] (nret);
		\node [tsb, below of=sadj] (vadj) {Ajuste da velocidade} edge [pre] (sadj);
		\node [tsb, below of=vadj] (act) {Envia as a��es} edge [pre] (vadj);
		\node [tsb, below of=act] (end) {Loop} edge [pre] (act);
		\node [8051b, right=2cm of sync] (rcv) {Recebe SYNC} edge [pre, dashed] (sync);
		\node [8051b, below of=rcv] (spkg) {Envia pacote} 
			edge [pre] (rcv)
			edge [post, dashed] (read);
		\node [8051b, right=2cm of act] (pwm) {Ajuste dos PWMs} edge [pre, dashed] (act);
		\path [line] (end) -- +(-2.5,0) |- (sync);
		% Background
		\begin{pgfonlayer}{background}
			\node [background, fit=(opentty) (end), label=above:tslogic] {};
			\node [background, fit=(nrcv) (nret), label=above:Fuzzy/FCM] {};
			\node [background, fit=(rcv) (pwm), label=above:C8051F340] {};
		\end{pgfonlayer}
	\end{tikzpicture}
	\caption[Diagrama em blocos software TS-7260]
	{Diagrama em blocos demonstrando o funcionamento do software da placa TS-7260.}
\end{figure}

\section{Algoritmos de Navega��o}
\label{sec:algs}
Esta se��o dedica-se � descri��o da implementa��o dos algoritmos de navega��o conforme descri��o te�rica dos mesmos apresentada na fundamenta��o te�rica. Mais espec�ficamente, ser�o descritos os c�digos respons�veis pela tomada de decis�es de ambos os algoritmos de navega��o que executam na placa TS-7260, o algoritmo de navega��o de l�gica Fuzzy simples e o ED-FCM.

\subsection{Algoritmo de Navega��o Fuzzy}
\label{sec:algfuzzy}
%Descri��o e referencia b�sica da biblioteca do fabro, adapta��o e utiliza��o. Descri��o das regras. INCOMPLETO
O algoritmo de Navega��o Fuzzy, baseado em infer�ncias sobre conjuntos Fuzzy, foi desenvolvido utilizando-se a biblioteca FLIE, desenvolvida originalmente pelo professor Jo�o Alberto Fabro. A FLIE, ou \emph{Fuzzy Logic Inference Engine}, � uma biblioteca que j� implementa um ambiente para a defini��o de conjuntos fuzzy e regras de infer�ncia sobre os mesmos, proporcionando � equipe mais tempo para o aprimoramento das regras e conjuntos e o teste dos mesmos. Este algoritmo ser� utilizado como base de compara��o para determina��o da efici�ncia do ED-FCM, descrito na se��o \ref{sec:algedfcm}. Nesta se��o, ser� descrito como a biblioteca FLIE foi utilizada neste projeto e como foram definidos os conjuntos fuzzy e as regras de infer�ncia.

\subsubsection{Vari�veis Lingu�sticas}

A biblioteca FLIE possui estruturas de dados, ou classes, prontas para serem utilizadas para defini��o de vari�veis lingu�sticas para a constru��o de um conjunto de vari�veis fuzzy, bem como estruturas para a defini��o de regras de infer�ncia baseadas neste conjunto de vari�veis fuzzy.
A engine permite a defini��o de vari�veis lingu�sticas na classe \emph{linguisticvariable}, que pode ser classificada atrav�s de valores fuzzy, com a classe \emph{category}. Cada \emph{category} � definida por uma faixa de valores v�lidos (\emph{range}), por 4 pontos que definem um trap�zio com os intervalos de subida, m�ximo e descida do n�vel de pertin�ncia do conjunto de entrada, para fuzzifica��o dos dados, e um nome apropriado, tal como Perto, Longe, R�pido, dependendo do contexto. Ou seja, cada valor de entrada � fuzzificado de acordo com um n�vel de pertin�ncia a cada categoria pertencente � vari�vel lingu�stica. A figura \ref{lingvar}, apresentada a seguir, demonstra de forma gr�fica esta estrutura.

\begin{figure}[!htb]
  \centering
  \includegraphics[width=\textwidth]{./figs/lingvar.png}
  \caption[Estrutura de uma Vari�vel Lingu�stica no FLIE.]
  {Estrutura de uma Vari�vel Lingu�stica no FLIE.}
  \label{lingvar}
\end{figure}

Utilizando esta estrutura, foram desenvolvidas tr�s vari�veis lingu�sticas para a entrada do sistema e duas vari�veis lingu�sticas para a sa�da do sistema. As vari�veis lingu�sticas de entrada correspondem aos cinco sensores do rob�... %testes..

\subsubsection{Regras de Infer�ncia}

Ap�s definidas as vari�veis lingu�sticas, regras de infer�ncia podem ser elaboradas atrav�s da classe \emph{infrule}. Esta classe � composta de, no m�ximo, tr�s vari�veis lingu�sticas de entrada e uma vari�vel lingu�stica de sa�da, todas definidas pela classe \emph{linguisticvariable}, seguindo a estrutura na figura \ref{infrule}.

\begin{figure}[!htb]
  \centering
  \includegraphics[width=\textwidth]{./figs/infrule.png}
  \caption[Estrutura de uma regra de infer�ncia no FLIE.]
  {Estrutura de uma regra de infer�ncia no FLIE.}
  \label{infrule}
\end{figure}

Com as vari�veis lingu�sticas de entrada e sa�da pode-se criar regras de infer�ncia. Cada regra � composta por uma sequ�ncia de valores fuzzy, sendo o �ltimo valor da sequ�ncia a sa�da fuzzy desejada para uma entrada correspondente aos um, dois ou tr�s valores antecedentes. Considerando esta limita��o � tr�s vari�veis, ... %unifica��o dos sensores, defini��o das regras

Finalmente, as regras definidas podem ser utilizadas para realizar infer�ncias. A bilioteca FLIE recebe os dados de entrada n�o fuzzificados, os fuzzifica, atribuindo valores de pertin�ncia para cada categoria definida pelo desenvolvedor, determina o n�vel de ativa��o de cada regra de infer�ncia definida e defuzzifica o resultado final de forma transparente.

\subsection{Algoritmo de Navega��o Baseado em ED-FCM}
\label{sec:algedfcm}
%Descri��o da adapta��o do c�digo matlab, utiliza��o, conceitos.
Esta se��o tem por objetivo descrever o projeto e a implementa��o do algoritmo de navega��o baseado na abordage FCM, apresentada na se��o \ref{sec:fcm} da fundamenta��o te�rica. O projeto consistiu na defini��o dos conceitos de entrada, n�vel e sa�da, rela��es causais e respectivos pesos. A implementa��o foi desenvolvida em linguagem C++ e compilada para ser executada na placa TS-7260.

Primeiramente, definiu a entrada do sistema de navega��o, a qual correspondeu �s leituras dos sensores de dist�ncia. Desse modo, foram criados tr�s conceitos de entrada:

\begin{enumerate}
\item SD - Representa a leitura do sensor lateral esquerdo.
\item SF - Representa a leitura do sensor frontal.
\item SE - Representa a leitura do sensor lateral direito.
\end{enumerate}

Esses conceitos guardam o valor da leitura da dist�ncia normalizada na faixa de 0 a 1, atrav�s da equa��o \ref{eq:normaliza}, sendo x o valor absoluto da dist�ncia (cm), MIN o valor m�nimo suportado pelo sensor e MAX o valor m�ximo.

\begin{equation}
\label{eq:normaliza}
y=\frac {x-MIN} {MAX-MIN}
\end{equation}

Em seguida, foram determinados os conceitos de n�vel do FCM, os quais estabelecem as infer�ncias resultantes dos valores dos conceitos de entrada. Foram definidos quatro conceitos n�vel:

\begin{enumerate}
\item GDF - Representa a intensidade da decis�o de girar a roda direita para frente.
\item GDT - Representa a intensidade da decis�o de girar a roda direita para tr�s.
\item GEF - Representa a intensidade da decis�o de girar a roda esquerda para frente.
\item GET - Representa a intensidade da decis�o de girar a roda esquerda para tr�s.
\end{enumerate}

Os n�veis desses conceitos s�o ativados atrav�s de uma fun��o sigmoidal dependente dos valores dos conceitos de entrada. As equa��es \ref{eq:sig_frente} e \ref{eq:sig_tras} foram adaptadas da equa��o \ref{sigmoide}, apresentada na fundamenta��o te�rica, e operam em dom�nio (0, 1) e imagem (0, 1).

\begin{equation}
\label{eq:sig_frente}
f_1(x)=\frac{1} {1+e^{(-10x+4,5)}}
\end{equation}

\begin{equation}
\label{eq:sig_tras}
f_2(x)=1-\frac{1} {10+e^{(-10x+4,5)}}
\end{equation}

Por fim, definiu-se a sa�da do sistema de navega��o, a qual s�o os n�veis percentuais de pot�ncia de cada roda. Esse n�vel varia na faixa de -100\% a +100\% e o sinal indica o sentido de giro, sendo o sinal positivo "girar para frente" e o sinal negativo "girar para tr�s". Desse modo, foram estabelecidos dois conceitos de sa�da:

\begin{enumerate}
\item RD Out - Representa o n�vel percentual de pot�ncia da roda direita.
\item RE Out - Representa o n�vel percentual de pot�ncia da roda esquerda.
\end{enumerate}

\begin{table}[htb!]
	\centering
	\caption[Rela��es causais do controlador FCM proposto]{Rela��es causais do controlador FCM proposto.}		
	\begin{tabular}[!htb]{ l l l l }
	  \hline
	  Rela��o causal & Descri��o & Efeito & Intensidade \\
	  \hline
		$w_1$ & SD influencia GDF & Positivo & Forte \\
		$w_2$ & SD influencia GEF & Negativo & Fraca \\
		$w_3$ & SD influencia GET & Positivo & Forte \\
		
		$w_4$ & SE influencia GEF & Positivo & Forte \\
		$w_5$ & SE influencia GDF & Negativo & Fraca \\
		$w_6$ & SE influencia GDT & Positivo & Forte \\
		
		$w_7$ & SF influencia GDF & Negativo & Fraca \\
		$w_8$ & SF influencia GDT & Positivo & M�dia \\
		$w_9$ & SF influencia GEF & Negativo & Fraca \\
		$w_{10}$ & SF influencia GET & Positivo & M�dia \\
		
		$w_{11}$ & GDF influencia RD Out & Positivo & Forte \\
		$w_{12}$ & GDT influencia RD Out & Negativo & Forte \\
		$w_{13}$ & GEF influencia RE Out & Positivo & Forte \\
		$w_{14}$ & GET influencia RE Out & Negativo & Forte \\
	  \hline  	
	\end{tabular}
	\label{tab:fcm-relacoes}
\end{table}

\begin{table}[htb!]
	\centering
	\caption[Valor num�rico dos pesos]{Valor num�rico dos pesos.}		
	\begin{tabular}[!htb]{ l l l l }
	  \hline
	  Intensidade & Valor num�rico \\
	  \hline
		FRACA & 0,125 \\
		M�DIA & 0,5 \\
		FORTE & 1,0 \\
	  \hline  	
	\end{tabular}
	\label{tab:pesos}
\end{table}

\begin{figure}[!htb]
    \centering
    \includegraphics[scale=0.70]{./figs/fcm_proposto.png}
    \caption[FCM proposto]{FCM proposto.}
    \fonte{\cite{FCMENDONCA}}
    \label{fig:fcm-proposto}
\end{figure}

Os conceitos foram interligados atrav�s das rela��es causais apresentadas na tabela \ref{tab:fcm-relacoes}, obtendo-se o mapa da figura \ref{fig:fcm-proposto}, e o valor num�rico dos pesos est� de acordo com a tabela \ref{tab:pesos}. As equa��es \ref{eq:gdf} a \ref{eq:get} determinam as intensidades dos conceitos de n�vel e as equa��es \ref{eq:rd_out} e \ref{eq:re_out} estabelecem a sa�da do FCM.

\begin{equation}
\label{eq:gdf}
GDF=\frac {w_1f_2(SD)+w_5f_1(SE)-w_7f_2(SF)} {w_1+w_5}
\end{equation}

\begin{equation}
\label{eq:gdt}
GDT=\frac {w_6f_2(SE)+w_8f_2(SF)} {w_6+w_8}
\end{equation}

\begin{equation}
\label{eq:gef}
GEF=\frac {w_4f_2(SE)+w_2f_1(SD)-w_9f_2(SF)} {w_4+w_2}
\end{equation}

\begin{equation}
\label{eq:get}
GET=\frac {w_3f_2(SE)+w_{10}f_2(SF)} {w_3+w_{10}}
\end{equation}

\begin{equation}
\label{eq:rd_out}
RD_{Out}=100 \times \frac {w_{11}GDF-w_{12}GDT} {FORTE}
\end{equation}

\begin{equation}
\label{eq:re_out}
RD_{Out}=100 \times \frac {w_{13}GEF-w_{14}GET} {FORTE}
\end{equation}

\section{Considera��es}
Nesta se��o foram descritos os passos da equipe desde os testes iniciais com o rob� at� o projeto da nova placa de roteamento. Com a realiza��o destas tarefas, o objetivo de reconstru��o da camada de baixo n�vel do rob� Bellator foi alcan�ado.


\chapter{Conclus�o}

O presente trabalho de conclus�o de curso apresentou objetivos abrangentes envolvendo desenvolvimento de hardware e software. No contexto da navega��o rob�tica, surgiu a necessidade de se utilizar um rob� real com a finalidade de se obterem resultados mais significativos. Desse modo, foi elaborado um projeto cujo escopo foi reconstruir e adequar uma plataforma previamente dispon�vel, que � descrita na sec��o \ref{sec:estpro}, por�m sem condi��es de uso imediato. A equipe procedeu com testes em laborat�rio de eletr�nica afim de avaliar as condi��es iniciais do rob�, conforme � descrito na sec��o \ref{sec:testecomp}. Uma an�lise de software foi efetuada e o c�digo original do microcontrolador C8051F340DK, disponibilizado como parte integrante do rob�, foi avaliado e reconfigurado de acordo com as necessidades do projeto, o que � descrito em detalhes na se��o \ref{sec:codmicro}. Havendo necessidade de implementa��o de hardware, a equipe projetou e construiu uma placa de roteamento para alimentar os sensores e encoders, assim como tratar os sinais destes e os de PWM, como � descrito na sec��o \ref{sec:desroteamento}. Finalmente, o hardware acoplado foi configurado e o software que executou os algortimos de navega��o foi desenvolvido, conforme a sec��o {sec:ts}. Com isso, a equipe concluiu a primeira parte do projeto, a qual consistiu na reconstru��o e adequa��o do rob� Bellator.

Estando a plataforma rob�tica funcional, procedeu-se ao projeto e implementa��o dos algoritmos de navega��o propostos. A L�gica Fuzzy foi estudada e apresentada na sec��o \ref{sec:logfuzzy} e a metodologia de Mapas Cognitivos Fuzzy (FCM) foi estudada e apresentada na sec��o \ref{sec:fcm}. Com a teoria fundamentada, a equipe projetou os algoritmos e implementou estes para executar na plataforma do projeto. Os c�digos foram escritos na linguagem C++ e compilados para executar no hardware da TS-7260. O projeto e a implementa��o dos algoritmos de navega��o fuzzy e FCM s�o descritos em detalhes nas sec��es \ref{sec:sec:algfuzzy} e \ref{sec:algedfcm}, respectivamente. Ap�s a implementa��o, a equipe submeteu os algoritmos a uma s�rie de testes b�sicos, denominada Testes Iniciais, que serviram como uma realimenta��o inicial do projeto dos algoritmos e pode ser lido na sec��o \ref{sec:testesini}. Finalizados esses testes, foram elaborados testes complexos, denominados Testes Avan�ados, para estressar os sistemas de navega��o propostos e fornecer outra realimenta��o do projeto dos algoritmos, conforme � descri��o na sec��o \ref{sec:testesavan}. Finalmente, ap�s os testes avan�ados, os algoritmos resolviam problemas complexos de navega��o, como o circuito em ``U" e o problema de decis�o quando dois obst�culos laterais e um frontal era colocado diante do rob�, e poderiam ser usados nos testes finais, que forneceram os dados para a an�lise de resultados e foram denominados Testes Comparativos, conforme � descrito na sec��o \ref{sec:testescomp}. Com isso foi conclu�do outros dois objetivos do projeto, que foram o projeto e implementa��o dos algoritmos de navega��o e a elabora��o e execu��o de uma metodologia de testes comparativos entre os algoritmos.

Para trabalhos futuros utilizando a plataforma Bellator reconstru�da, a equipe recomenda combinar sensores de ultra-som ao sensores infra vermelho, justificando isso porque os sensores de ultra-som apresentam uma faixa de opera��o cuja dist�ncia m�nima � menor que a do infra-vermelho, podendo capturar dist�ncia 2 cm. Desse modo, os algoritmos poderiam operar em uma faixa mais abrangente. Atualmente a dist�ncia m�nima suportada pelo sistema de navega��o � de 15 cm. Outra sugest�o � introduzir ao sistema uma realimenta��o por b�ssola pois nesse projeto a realimenta��o odom�trica fornecida pelos encoders � utilizada para ajustar a velocidade das rodas e n�o faz uma interpreta��o da dire��o do rob�. Para tornar o rob� seguro para o manuseio, sugere-se a fixa��o dos sensores parafusando-os no chassi do Bellator e acoplar uma casca que proteja os circuitos microcontrolados. A equipe tamb�m remenda a reconstru��o da placa de roteamento utilizando um m�todo industrial para confecc��o de placas de circuito impresso. Para os sistemas de navega��o, um trabalho futuro de grande riqueza seria introduzir ao sistema a capacidade de interpretar a posi��o do rob� em rela��o a um referencial. Com isso, o rob� seria capaz de resolver problemas nos quais este deve partir de um ponto inicial no espa�o a um ponto final, guiando-se pelos sensores para evitar colis�es e realimentar-se por um sistema de posicionamento para corrigir a traget�ria. Outro trabalho, produto deste, seria introduzir ao sistema uma mem�ria a qual pudesse mapear os obst�culos capturados pelos sensores do rob�, assim sendo, produzir-se-ia um artefato aut�nomo capaz de mapear terrenos.

Por fim, a equipe concluiu esta monografia justificando que os objetivos descritos na introdu��o, sec��o \ref{rec:obj}, foram alcan�ados e est�o de acordo com os requisitos m�nimos de um curso de Engenharia de Computa��o. Os problemas encontrados na execu��o do projeto est�o associados ao escopo abrangente do mesmo, o qual envolveu o desenvolvimento de hardware e software em um projeto integrado. A subdivis�o do projeto em diversos objetivos, sendo que um foi pr�-requisito para a execu��o do outro em uma execu��o encadeada, foi inevit�vel para alcan�ar os resultados finais. A equipe encontrou dificuldades em todas as etapas do projeto, desde a reconstru��o do rob� at� os testes avan�ados e an�lise de resultados. Os componentes eletr�nicos foram testados isoladamente e houve o riscos de haver danos, o que representaria atrasos no projeto. A placa TS-7260 apresentou certa complexidade para ser configurada pois n�o havia um t�cnico dispon�vel para auxiliar a equipe, a qual teve que aprender a trabalhar com esse hardware. Os testes de integra��o da C8051F340DK e da TS-7260, operacionalizando o rob�, exigiram processos de depura��o integrados, nos quais os problemas foram isolados e corrigidos repetidas vezes. A implementa��o dos algoritmos at� a vers�o final foi realizada paralelamente aos testes b�sicos e avan�ados, nas quais os problemas de navega��o foram detectados, isolados e corrigidos. A metodologia de testes escolhida foi elaborada pela equipe e foram efetuados v�rios experimentos com registro em v�deo at� que se atingessem os resultados finais. Tendo com base os v�deos gravados e a experi�ncia em campo observada, a equipe precisou analisar os resultados, discut�-los e extrair conclus�es para finalizar o projeto, sendo que essa tarefa representou um trabalho cient�fico.


%---------- Referencias ----------
\bibliography{reflatex} % geracao automatica das referencias a partir do arquivo reflatex.bib

\apendice
\chapter{Refer�ncia de Montagem do Rob�}

A figura \ref{montagembellator} � uma composi��o de partes de fotos do rob� Bellator em seu est�gio final de montagem, ao final do projeto. A alimenta��o do motor em "2", bem como a liga��o dos mesmos atrav�s da chave indicada abaixo de "9" na vis�o geral do rob�, e a liga��o dos outros componentes em "3" s� dever�o ser realizados ap�s verificar todas as outras conex�es do rob� conforme indicadas na figura. Ligar incorretamente alguma das conex�es pode levar � queima de componentes. Os cabos \emph{flat} devem ser conectados de modo a alinhar o \emph{GND} da placa de roteamento com o \emph{GND} da C8051F340 no \emph{port} correspondente. Tamb�m deve-se ficar atento ao n�vel de carga das baterias, que dever� ser de 10Volts ou mais. Ap�s realizadas todas as conex�es, a TS ir� inicializar (cerca de 5 minutos) e poder� ser, enfim, utilizada conforme documenta��o em \ref{sec:ts}. A lista a seguir relaciona cada item numerado na figura com sua respectiva documenta��o ao longo do documento.

\begin{itemize}
\item 1 e 3) Especifica��o da C8051F340DK em \ref{sec:C8051F340DK}, TS 7260 em \ref{sec:espts7260} e placa de roteamento em \ref{sec:espplacaroteamento}.
\item 2) Especifica��o dos motores em \ref{sec:robobellator}.
\item 4 a 9) Diagramas esquem�ticos e \emph{interface} da placa de roteamento com a C8051F340 em \ref{sec:desroteamento}.
\item Especifica��o dos Sensores e Encoders em \ref{sec:sensores} e \ref{sec:heds9700}.
\item 10) O pendrive cont�m o c�digo executado na TS7260, descrito em \ref{sec:ts}.
\item 11) As pontes H s�o alimentadas pelos conectores de alimenta��o do motor.
\end{itemize}

\begin{figure}[H]
  \centering
  \includegraphics[width=0.8\textwidth]{./figs/Montagem_Bellator.png}
  \caption[Refer�ncia de Montagem do Rob�]
  {Refer�ncia de Montagem do Rob�.}
  \label{montagembellator}
  \fonte{Autoria pr�pria}
\end{figure}

\end{document}

